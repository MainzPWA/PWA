\documentclass[a4paper,10pt]{article}
\usepackage{afterpage}
\usepackage{amssymb}
\usepackage{amsmath}
\usepackage{color}
\usepackage{enumerate}
\usepackage[latin1]{inputenc}
%\usepackage{times}
%\usepackage{mathptmx}
%\usepackage{t1enc}
%\usepackage{german}
\usepackage{graphicx}
\usepackage{latexsym}
\usepackage{mathrsfs}
%\usepackage{oldgerm}
\usepackage{pifont}
\usepackage{psfrag}
\usepackage{rotating}
%\usepackage{showkeys}
\usepackage[german]{varioref}
\usepackage{wrapfig}
\usepackage{wasysym}
%\usepackage{txfonts}
\usepackage{ulem}
\usepackage[T1]{fontenc} 
%\usepackage[helvet]{sfmath}
%\usepackage{helvet}
\usepackage{eurosym}
\usepackage{geometry}
\geometry{left=20mm,right=20mm, top=2cm, bottom=2cm} 
\parindent=0cm

\def\PWA{\ttfamily PWA\rmfamily\ }
\def\d{\mathrm{d}}
\def\cc{^*}
\def\Re{\mathrm{Re}}
\def\Im{\mathrm{Im}}
\def\enn{\mbox{\ttfamily\textit{n}\rmfamily}}
\def\bl{\phantom{0}}
\def\tt{\ttfamily}
\def\rm{\rmfamily}
\renewcommand{\arraystretch}{1.5}

% Title Page
\title{Using the Mainz Multipole Fitter \PWA}
\author{Sven Schumann}

\begin{document}
\maketitle

\begin{abstract}

\end{abstract}


\section{Introduction}

\PWA is a single energy (SE) multipole fitting tool, used to determine electromagnetic multipoles for pseudoscalar
meson photoproduction ($\pi$, $\eta$, $K$, ...) from experimental cross sections (polarised and unpolarised)
and/or asymmetry observables. Multipoles are extracted using a standard $\chi^2$ minimisation scheme with the
possibility to add additional constraints imposed by existing partial wave analyses and models for meson photoproduction
(MAID, SAID, BnGa, ...). These constraints are implemented using so-called penalty terms, giving an additional
contribution in the minimisation process.

\section{$\chi^2$ minimisation and multipole extraction}

\PWA determines electromagnetic multipoles as fit parameters in a $\chi^2$ minimisation process with respect to
experimental photoproduction results in form of the standard photoproduction observables, which are typically
given in four groups
\begin{displaymath}
\begin{array}{ll}
 \mbox{Group $S$:} &\frac{\d\sigma}{\d\Omega} = \sigma_0, \Sigma, T, P\\
 \mbox{Group $BT$:}& E, F, G, H\\
 \mbox{Group $BR$:}& C_{x^\prime}, C_{z^\prime}, O_{x^\prime}, O_{z^\prime}\\
 \mbox{Group $TR$:}& T_{x^\prime}, T_{z^\prime}, L_{x^\prime}, L_{z^\prime}
\end{array}
\end{displaymath}
Right now, \PWA only supports observable data for groups $S$, $BT$, and $BR$. These observables can be expressed
in form of the four CGLN amplitudes $F_i$ according to
\begin{displaymath}\label{frm_Obs}
\begin{array}{lll}
\sigma_0 & = & \Re \left\lbrace |F_1|^2 + |F_2|^2 - 2\cos\theta F_1\cc F_2  + \frac{1}{2}\sin^2\theta\left[ |F_3|^2 + |F_4|^2 + 2 F_2\cc  F_3    + 2 F_1\cc  F4 + 2\cos\theta F_3\cc F_4\right] \right\rbrace \cdot \rho\\
\sigma_0 \Sigma& = & - \frac{1}{2}\sin^2\theta\:\Re\left\lbrace|F_3|^2 + |F_4|^2 
                                               + 2\left[F_2\cc F_3 + F_1\cc F_4 
                                               + \cos\theta F_3\cc F_4\right]  
                                               \right\rbrace \cdot \rho\\
\sigma_0 T     & = & \sin\theta\:\Im \left\lbrace F_1\cc F_3 - F_2\cc F_4 
                                               + \cos\theta\left(F_1\cc F_4 - F_2\cc F_3\right)
                                               - \sin^2\theta F_3\cc F_4
                                               \right\rbrace \cdot \rho\\
\sigma_0 P     & = & \sin\theta\: \Im\left\lbrace F_2\cc F_4 - 2 F_1\cc F_2 - F_1 F_3
                                               + \cos\theta\left(F_2\cc F_3 - F_1\cc F_4\right) 
                                               + \sin^2\theta F_3\cc F_4
                                               \right\rbrace \cdot \rho\\
\sigma_0 E     & = & \Re\left\lbrace|F_1|^2 + |F_2|^2 - 2\cos\theta F_1\cc F_2
                                               +\sin^2\theta\left(F_2\cc F_3 + F_1\cc F_4\right)
                                               \right\rbrace \cdot \rho\\
\sigma_0 F     & = & \sin\theta \:\Re\left\lbrace F_1\cc F_3 - F_2\cc F_4 
                                               + \cos\theta\left(F_1\cc F_4 + F_2\cc F_3\right)
                                                 \right\rbrace \cdot \rho\\
\sigma_0 G     & = & \sin^2\theta \:\Im\left\lbrace F_2\cc F_3 + F_1\cc F_4\:
                                                \right\rbrace \cdot \rho\\
\sigma_0 H\,     & = & \sin\theta \:\Im \left\lbrace 2 F_1\cc F_2 + F_1\cc F_3 
                                               - F_2\cc F_4 -\cos\theta\left( F_2\cc F_3 - F_1\cc F_4 \right)
                                                 \right\rbrace \cdot \rho\\
\sigma_0 O_{x^\prime} & = & -\sin\theta \:\Im \left\lbrace F_1\cc F_4 - F_2\cc F_3
                                                      + \cos\theta\left(F_1\cc F_3 - F_2\cc F_4\right)
                                                       \right\rbrace \cdot \rho\\
\sigma_0 O_{z^\prime} & = & -\sin^2\theta \:\Im \left\lbrace F_1\cc F_3 + F_2\cc F_4\:
                                                       \right\rbrace \cdot \rho\\
\sigma_0 C_{x^\prime} & = & \sin\theta \:\Re \left\lbrace |F_1|^2 - |F_2|^2 
                                                      + F_1\cc F_3 - F_2\cc F_3 
                                                      +\cos\theta\left(F_1\cc F_3 - F_2\cc F_4\right)
                                                       \right\rbrace \cdot \rho\\
\sigma_0 C_{z^\prime} & = & \Re \left\lbrace 2 F_1\cc F_2 
                                                      +\sin^2\theta\left(F_1\cc F_3 + F_2\cc F_4\right)
                                                      -\cos\theta\left(|F_1|^2+|F_2|^2\right)
                                                       \right\rbrace \cdot \rho\\
\end{array}
\end{displaymath}
with a phase space factor $\rho = \frac{q}{k}$ given by pion and photon momenta $q$ and $k$. The CGLN amplitudes
depend on the electromagnetic multipoles $E_{L\pm}$, $M_{L\pm}$ for multipole orders $L$ up to $L_\mathrm{max}$
\begin{displaymath}\label{frm_CGLN}
\begin{array}{l}
F_1 = \sum\limits_{L\geq 0}^{L_\mathrm{max}} \left\lbrace (L\cdot M_{L+} + E_{L+})P^{\prime}_{L+1} 
+ \left[(L+1)\cdot M_{L-} + E_{L-}\phantom{{}^{\prime}_{L}}\right]P^{\prime}_{L-1} \right\rbrace\\
F_2 = \sum\limits_{L\geq 1}^{L_\mathrm{max}} \left[(L+1)\cdot M_{L+} + L\cdot M_{L-} \phantom{{}^{\prime}_{L}}  \right]P^{\prime}_{L}\\
F_3 = \sum\limits_{L\geq 1}^{L_\mathrm{max}} \left[(E_{L+} - M_{L+})P^{\prime\prime}_{L+1}+ (E_{L-} + M_{L-})P^{\prime\prime}_{L-1} \right]\\
F_4 = \sum\limits_{L\geq 2}^{L_\mathrm{max}}\left( M_{L+} - E_{L+} - M_{L-} - E_{L-}\right)P^{\prime\prime}_{L}
\end{array}
\end{displaymath}
and derivatives of the Legendre polynoms $P_L$ describing the angular dependencies. Using these relations between observables
and electromagnetic multipoles, it is possible to create a $\chi^2$ function having multipole as fit parameters and physical
observables as experimental input
\begin{displaymath}
\begin{array}{lcl}
\chi^2
&=& \sum\limits_i \left(
\frac
{\sigma_0^\mathrm{exp}(\theta_i) - \sigma_0^\mathrm{fit}(\theta_i)}
{\Delta \sigma_0^\mathrm{exp}(\theta_i)}
\right)^2\\
&+& \sum\limits_i \left(
\frac
{\Sigma^\mathrm{exp}(\theta_i) - \Sigma^\mathrm{fit}(\theta_i)}
{\Delta \Sigma^\mathrm{exp}(\theta_i)}
\right)^2
+ \sum\limits_i \left(
\frac
{\sigma_\Sigma^\mathrm{exp}(\theta_i) - \sigma_\Sigma^\mathrm{fit}(\theta_i)}
{\Delta \sigma_\Sigma^\mathrm{exp}(\theta_i)}
\right)^2\\
&+& \sum\limits_i \left(
\frac
{T^\mathrm{exp}(\theta_i) - T^\mathrm{fit}(\theta_i)}
{\Delta T^\mathrm{exp}(\theta_i)}
\right)^2
+ \sum\limits_i \left(
\frac
{\sigma_T^\mathrm{exp}(\theta_i) - \sigma_T^\mathrm{fit}(\theta_i)}
{\Delta \sigma_T^\mathrm{exp}(\theta_i)}
\right)^2\\
&+& \sum\limits_i \left(
\frac
{P^\mathrm{exp}(\theta_i) - P^\mathrm{fit}(\theta_i)}
{\Delta P^\mathrm{exp}(\theta_i)}
\right)^2
+ \sum\limits_i \left(
\frac
{\sigma_P^\mathrm{exp}(\theta_i) - \sigma_P^\mathrm{fit}(\theta_i)}
{\Delta \sigma_P^\mathrm{exp}(\theta_i)}
\right)^2\\
&+& \sum\limits_i \left(
\frac
{E^\mathrm{exp}(\theta_i) - E^\mathrm{fit}(\theta_i)}
{\Delta E^\mathrm{exp}(\theta_i)}
\right)^2
+ \sum\limits_i \left(
\frac
{\sigma_E^\mathrm{exp}(\theta_i) - \sigma_E^\mathrm{fit}(\theta_i)}
{\Delta \sigma_E^\mathrm{exp}(\theta_i)}
\right)^2\\
&+& \sum\limits_i \left(
\frac
{F^\mathrm{exp}(\theta_i) - F^\mathrm{fit}(\theta_i)}
{\Delta F^\mathrm{exp}(\theta_i)}
\right)^2
+ \sum\limits_i \left(
\frac
{\sigma_F^\mathrm{exp}(\theta_i) - \sigma_F^\mathrm{fit}(\theta_i)}
{\Delta \sigma_F^\mathrm{exp}(\theta_i)}
\right)^2\\
&+& \sum\limits_i \left(
\frac
{G^\mathrm{exp}(\theta_i) - G^\mathrm{fit}(\theta_i)}
{\Delta G^\mathrm{exp}(\theta_i)}
\right)^2
+ \sum\limits_i \left(
\frac
{\sigma_G^\mathrm{exp}(\theta_i) - \sigma_G^\mathrm{fit}(\theta_i)}
{\Delta \sigma_G^\mathrm{exp}(\theta_i)}
\right)^2\\
&+& \sum\limits_i \left(
\frac
{H^\mathrm{exp}(\theta_i) - H^\mathrm{fit}(\theta_i)}
{\Delta H^\mathrm{exp}(\theta_i)}
\right)^2
+ \sum\limits_i \left(
\frac
{\sigma_H^\mathrm{exp}(\theta_i) - \sigma_H^\mathrm{fit}(\theta_i)}
{\Delta \sigma_H^\mathrm{exp}(\theta_i)}
\right)^2
\end{array}
\end{displaymath}
where $\sigma_X^\mathrm{exp}(\theta_i)$, $X^\mathrm{exp}(\theta_i)$ describe experimental cross sections and asymmetries
at their polar angle positions $\theta_i$, while 
$\sigma_X^\mathrm{fit}(\theta_i)$, $X^\mathrm{fit}(\theta_i)$ are cross sections and asymmetries as calculated from
multipole parameters $E_{L\pm}$, $M_{L\pm}$.
Minimisation of the above $\chi^2$ function will therefore adjust electromagnetic multipole values n order for the best describtion
of available data.
\PWA will use asymmetries $X$ as well as polarised cross sections $\sigma_x = \sigma_0\cdot X$ as experimental input,
so there is no need to transform one type of experimental information to another beforehand.

\section{Penalty terms}

\subsection{Penalty mode \textit{MLP1}}
\begin{displaymath}
\begin{array}{lcl}
 Q_1 &=& \dfrac{q_1}{|\mathcal{M}^\mathrm{sol}|^2}\cdot \dfrac{N_\mathrm{pts}}{N_\mathrm{par}} \cdot \left[
\sum\limits_L^{L_\mathrm{max}}\left(\Re E_{\ell-}^\mathrm{fit} - \Re E_{\ell-}^\mathrm{sol}\right)^2
+
\sum\limits_L^{L_\mathrm{max}}\left(\Im E_{\ell-}^\mathrm{fit} - \Im E_{\ell-}^\mathrm{sol}\right)^2\right.\\
&&\hspace{6.73em}+
\sum\limits_L^{L_\mathrm{max}}\left(\Re E_{\ell+}^\mathrm{fit} - \Re E_{\ell+}^\mathrm{sol}\right)^2
+
\sum\limits_L^{L_\mathrm{max}}\left(\Im E_{\ell+}^\mathrm{fit} - \Im E_{\ell+}^\mathrm{sol}\right)^2\\
&&\hspace{6.73em}+
\sum\limits_L^{L_\mathrm{max}}\left(\Re M_{\ell-}^\mathrm{fit} - \Re M_{\ell-}^\mathrm{sol}\right)^2
+
\sum\limits_L^{L_\mathrm{max}}\left(\Im M_{\ell-}^\mathrm{fit} - \Im M_{\ell-}^\mathrm{sol}\right)^2\\
&&\hspace{6.73em}+\left.
\sum\limits_L^{L_\mathrm{max}}\left(\Re M_{\ell+}^\mathrm{fit} - \Re M_{\ell+}^\mathrm{sol}\right)^2
+
\sum\limits_L^{L_\mathrm{max}}\left(\Im M_{\ell+}^\mathrm{fit} - \Im M_{\ell+}^\mathrm{sol}\right)^2
\right]
\end{array}
\end{displaymath}

\subsection{Penalty mode \textit{MLP2}}

\begin{displaymath}
\begin{array}{lcl}
 Q_2 &=& q_2\cdot \dfrac{N_\mathrm{pts}}{N_\mathrm{par}} \cdot \left[
\sum\limits_L^{L_\mathrm{max}}
\frac{\left(\Re E_{\ell-}^\mathrm{fit} - \Re E_{\ell-}^\mathrm{sol}\right)^2}
{\left(\Re E_{\ell-}^\mathrm{sol}\right)^2 + \varepsilon^2}
+
\sum\limits_L^{L_\mathrm{max}}
\frac{\left(\Im E_{\ell-}^\mathrm{fit} - \Im E_{\ell-}^\mathrm{sol}\right)^2}
{\left(\Im E_{\ell-}^\mathrm{sol}\right)^2 + \varepsilon^2}\right.\\
&&\hspace{4.19em}+
\sum\limits_L^{L_\mathrm{max}}
\frac{\left(\Re E_{\ell+}^\mathrm{fit} - \Re E_{\ell+}^\mathrm{sol}\right)^2}
{\left(\Re E_{\ell+}^\mathrm{sol}\right)^2 + \varepsilon^2}
+
\sum\limits_L^{L_\mathrm{max}}
\frac{\left(\Im E_{\ell+}^\mathrm{fit} - \Im E_{\ell+}^\mathrm{sol}\right)^2}
{\left(\Im E_{\ell+}^\mathrm{sol}\right)^2 + \varepsilon^2}\\
&&\hspace{4.19em}+
\sum\limits_L^{L_\mathrm{max}}
\frac{\left(\Re M_{\ell-}^\mathrm{fit} - \Re M_{\ell-}^\mathrm{sol}\right)^2}
{\left(\Re M_{\ell-}^\mathrm{sol}\right)^2 + \varepsilon^2}
+
\sum\limits_L^{L_\mathrm{max}}
\frac{\left(\Im M_{\ell-}^\mathrm{fit} - \Im E_{\ell-}^\mathrm{sol}\right)^2}
{\left(\Im M_{\ell-}^\mathrm{sol}\right)^2 + \varepsilon^2}\\
&&\hspace{4.19em}+\left.
\sum\limits_L^{L_\mathrm{max}}
\frac{\left(\Re M_{\ell+}^\mathrm{fit} - \Re M_{\ell+}^\mathrm{sol}\right)^2}
{\left(\Re M_{\ell+}^\mathrm{sol}\right)^2 + \varepsilon^2}
+
\sum\limits_L^{L_\mathrm{max}}
\frac{\left(\Im M_{\ell+}^\mathrm{fit} - \Im E_{\ell+}^\mathrm{sol}\right)^2}
{\left(\Im M_{\ell+}^\mathrm{sol}\right)^2 + \varepsilon^2}
\right]
\end{array}
\end{displaymath}

\subsection{Penalty mode \textit{MLP3}}
\begin{displaymath}
Q_3 = Q_1 + Q_2
\end{displaymath}


\subsection{Penalty mode \textit{CGLN}}

\begin{displaymath}
 Q_4 =\sum_{i=1}^4
\sum_{j=1}^{N_\mathrm{pts}}
\frac{|F_i(x_j)^\mathrm{fit} - F_i(x_j)^\mathrm{sol}|^2}{f_i^2}
\end{displaymath}

\subsection{Penalty mode \textit{HELI}}

\begin{displaymath}
 Q_5 =\sum_{i=1}^4
\sum_{j=1}^{N_\mathrm{pts}}
\frac{|H_i(x_j)^\mathrm{fit} - H_i(x_j)^\mathrm{sol}|^2}{h_i^2}
\end{displaymath}
with
\begin{displaymath}\label{frm_Heli}
\begin{array}{l}
H_1 =           -\dfrac{1}{\sqrt{2}}\sin\theta\cos\frac{\theta}{2} \left(F_3 + F_4\right) \\
H_2 = \phantom{-}\sqrt{2}\cos\frac{\theta}{2}\left[\left(F_2 - F_1\right) + \frac{1-\cos\theta}{2}\left(F_3 - F_4\right)\right] \\
H_3 = \phantom{-}\dfrac{1}{\sqrt{2}}\sin\theta\sin\frac{\theta}{2} \left(F_3 - F_4\right) \\
H_4 = \phantom{-}\sqrt{2}\sin\frac{\theta}{2}\left[\left(F_1 + F_1\right) + \frac{1+\cos\theta}{2}\left(F_3 + F_4\right)\right]
\end{array}
\end{displaymath}

\section{Setting up a work directory for \PWA}

It is recommended to set up a special work directory for your \PWA SE fits. This directory should contain
all necessary experimental data, model inputs used for the fits, and a configuration file for \tt PWA\rm.
A typical work directory consists of the follwing entries:
\begin{itemize}
\item A directory \tt model\rm\ containing model values for electromagnetic multipoles, which are used
as start parameters and for penalty calculations during the fitting process.
It is recommended to have a \tt model\rm\ directory as a symbolic link to one of the models that \PWA provides within 
its package, e.g. use the command\\
\tt ln -s /home/\textit{username}/PWA/model/ppi0/MAID model\rm\\
to use MAID calculations for your $p \pi^0$ SE fits. 
As the model path has to be given in the \PWA configuration file, it is not required to have a dedicated \tt model\rm\ directory, but
not having absolute paths in the \PWA configuration file can simplify things.

\item A directory \tt data\rm\ containing experimental datasets for observables you want to use in your fit.
It is recommended to have a \tt data\rm\ directory as a symbolic link to the datasets that \PWA provides within 
its package, e.g. use the command\\
\tt ln -s /home/\textit{username}/PWA/data/ppi0 data\rm\\
This will make various $p\pi^0$ datasets accessible to your work directory. As filenames for all datasets you want to use for fits
have to be given in the \PWA configuration file, it is not required to have a dedicated \tt data\rm\ directory, but
not having absolute paths in the \PWA configuration file can simplify things.

\item A \tt PWA.cfg\rm\ file, holding various configuration options (see section \ref{sc_config}).
\PWA by default tries to read its configuration from \tt PWA.cfg\rm\ in the current directory, but a different filename
can be passed as command-line parameter, e.g.\\
\PWA \tt /home/\textit{username}/cfg/fit.cfg\rm

\item The \PWA executable. Here it is recommended to either create a (symbolic) link to the actual \PWA binary
which normally can be found under \tt build/bin/PWA\rm\ or to include \PWA  in your search path.
\end{itemize}
During the fits \PWA will create additional \tt plots.\textit{n}\rm\ directories holding the fit results.
The number of \tt plots.\textit{n}\rm\ directories depends on the amount of solutions you want \PWA to store.
An example of such a work directory for $p \pi^0$ data and fits based on MAID can be found in
\tt work/ppi0/MAID\rm\ in the main directory of the \PWA package.

\section{Configuration options}\label{sc_config}

The following sections describe the various configuration options for fits with \PWA.
Options are generally read from the configuration file \tt PWA.cfg\rm\ in your \PWA work directory, if \PWA
is invoked without additional command-line parameters. However, a different configuration file may be passed as parameter,
e.g.\\
\PWA \tt fit.cfg\rm\\
 All following configuration keywords are case-sensitive, which makes
\tt OPTION\rm\ different from \tt option\rm, where all keywords have to  be upper case.
Any non-existing keyword will be ignored, therefore no explicit markup for comments is required.

\subsection{Kinematic parameters}

\begin{itemize}
\item
\tt MASS\_MESON \textit{mass}\rm\\
Sets the mass of the produced final state meson ($\pi^0$, $\pi^\pm$, $\eta$, $\eta^\prime$, $K^\pm$, ...) to the given value
 (in MeV). This parameter is mandatory for correct calculation of kinematic factors during the fitting process.
Examples are
\vspace{-0.5em}\begin{itemize}
 \item[$\pi^0$\:]          $134.9766$
 \item[$\pi^+$]            $139.5702$
 \item[$\eta\phantom{^+}$] $547.8530$
\end{itemize}

\tt MASS\_INITIAL \textit{mass}\rm\\
Sets the mass of the inital state baryon ($p$, $n$) to the given value
 (in MeV). This parameter is mandatory for correct calculation of kinematic factors during the fitting process.
Examples are
\vspace{-0.5em}\begin{itemize}
 \item[$p$\:\:] $938.2720$
 \item[$n$\:\:] $939.5654$
\end{itemize}

\tt MASS\_FINAL \textit{mass}\rm\\
Sets the mass of the produced final state baryon ($p$, $n$, $\Lambda$, $\Sigma$, ...) to the given value
 (in MeV). This parameter is mandatory for correct calculation of kinematic factors during the fitting process.
Examples are
\vspace{-0.5em}\begin{itemize}
 \item[$p$\:\:] $\phantom{1}938.2720$
 \item[$n$\:\:] $\phantom{1}939.5654$
 \item[$\Lambda$\:\:]     $1115.6830$
\end{itemize}
\end{itemize}

\subsection{Fitted multipoles}

\begin{itemize}
\item
\tt L\_MAX \textit{n}\rm\\
This sets the maximum multipole order $L_\mathrm{max}$ (which is considered in the calculation of CGLN amplitudes and observables)
to the given value \tt\textit{n}\rm.
Note that this does \textbf{not} set the maximum multipole order which is actually fitted, as the fittable multipoles
can be selected indiviually later on. Any multipole with $L \leq L_\mathrm{max}$ that is not fitted will enter
the calculation of CGLN amplitudes and observables as a parametrisation using the current model values.

\item
\tt FIX\_E\textit{n}P 0/1\rm\\
\tt FIX\_M\textit{n}P 0/1\rm\\
\tt FIX\_E\textit{n}M 0/1\rm\\
\tt FIX\_M\textit{n}M 0/1\rm\\
Any $E_{n+}$, $M_{n+}$, $E_{n-}$, $M_{n-}$ multipole can either be fixed (\tt1\rm) to their model values
or be used as fit parameters (\tt0\rm).
Fixing a multipole here fixes both magnitude and phase (or real and imaginary values).
Hence any fittable multipole will add two parameters to the fitting process.

\item
\tt FIX\_E\textit{n}P\_PHASE 0/1\rm\\
\tt FIX\_E\textit{n}M\_PHASE 0/1\rm\\
\tt FIX\_M\textit{n}P\_PHASE 0/1\rm\\
\tt FIX\_M\textit{n}M\_PHASE 0/1\rm\\
For any $E_{n+}$, $M_{n+}$, $E_{n-}$, $M_{n-}$ multipole that was previously set as fittable the individual
phase can be either fixed (\tt1\rm) to its model value or be used as fit parameter (\tt0\rm). Fixing a phase
will reduce the number of fit parameters by one and only fit the magnitude for the given multipole.

\end{itemize}

\subsection{Penalty options}

\begin{itemize}
\item
\tt PENALTY\_MODE \textit{n}\rm\\
Selects a penalty mode for use during the minimisation process. Possible values for $\enn$ are
\vspace{-0.5em}\begin{enumerate}
\item[0\:\:] No penalty contribution
\item[1\:\:] \textit{MLP1}\hspace{0.16em} penalty mode
\item[2\:\:] \textit{MLP2}\hspace{0.16em} penalty mode
\item[3\:\:] \textit{MLP3}\hspace{0.17em} penalty mode
\item[4\:\:] \textit{CGLN} penalty mode
\item[5\:\:] \textit{HELI}\hspace{0.43em} penalty mode
\end{enumerate}

\item
\tt PENALTY\_MLP1 \textit{value}\rm\\
Sets the weight factor $q_1$ to the given value. This value will be used for for \textit{MLP1} and \textit{MLP3} penalty modes.
Typical values are around $q_1 = 2.0$.
\item
\tt PENALTY\_MLP2 \textit{value}\rm\\
Sets the weight factor $q_2$ to the given value. This value will be used for for \textit{MLP2} and \textit{MLP3} penalty modes.
Typical values are around $q_2 = 0.5$.

\item
\tt WEIGHT\_A\textit{n} \textit{value}\rm\\
Sets the individual weight factors for CGLN or helicity amplitude $\mbox{\tt\textit{n}\rm} = 1...4$ to the given value. These values will be used
in CGLN or HELI penalty mode only. Typical values are around $f_n = 1.0$.

\item
\tt PRINT\_PENALTY 0/1\rm\\
When enabled (\tt1\rm) an additional line will be printed out for each fit giving the individual contributions of $\chi^2$ and
penalty $Q$ for the found solution.

\end{itemize}

\subsection{Error calculation}

\begin{itemize}
\item
\tt ERROR\_MODE \textit{n}\rm\\
Selects a minimisation function for error calculation through MINUIT. Possible options for $\enn$ are
\vspace{-0.5em}\begin{enumerate}
\item[1\:\:] Use only $\chi^2$ function for error determination. This is the mathematically most sound approach, as the definition of
parameter error ranges is based on the variation of $\chi^2 \rightarrow \chi^2 +1$. However, for a
minimisation with penalty contributions, the minima for $\chi^2$ only and $\chi^2+Q$ are not necessarily
identical and the behaviour of $\chi^2$ only around the chosen minimum might result in ill-defined parameter errors.
\item[2\:\:] Use sum of $\chi^2$ and current penalty $Q$ for error determination. This might in general overestimate
parameter errors but will always produce well-defined results.
\item[3\:\:] Use an adaptive mode that prefers the $\chi^2$ only calculation but falls back to the 
sum of $\chi^2$ and current penalty $Q$ if this produces smaller errors. This is the recommended mode.
\end{enumerate}

\end{itemize}

\subsection{Multiple fit solutions}

\begin{itemize}
\item
\tt ITERATIONS \textit{n}\rm\\
Indiviual SE fits will be repeated $\enn$ times with small variatons in the starting parameters to find possible additional
minima in the parameter space. When the given number of iterations has been processed the best solution(s), i.e.
with lowest $\chi^2+Q$, are selected. Computimg time for the fitting process increases linear with the number of iterations.
For reasonable performance, $\enn$ should not exceed values of $50...100$.

\item
\tt SOLUTIONS \textit{n}\rm\\
Up to $\enn$ unique solutions that have been found during the iterations for each SE fit will be stored (with decreasing fit quality).
The number of maximum solutions to store may not exceed the number of iterations for the fit process.

\item
\tt VARIATION\_REL \textit{value}\rm\\
This will define a \textbf{relative} variation of start parameters for each iteration equally distributed
in a band around its current model value. The relative size of this band is given by $\pm$\tt\textit{value}\rm. A typical value
is $0.2$ which will result in parameter variations by $\pm20\%$.  Note that relative and absolute variations will be added.

\item
\tt VARIATION\_ABS \textit{value}\rm\\
This will define an \textbf{absolute} variation of start parameters for each iteration equally distributed
in a band around its current model value. The size of this band is given by $\pm$\tt\textit{value}\rm\ in units of $10^{-3}/m_{\pi^+}$.
This option is useful to define an absolute minimum of the variation band, which will be effective for very small
multipoles (where a relative variation would have only a small effect). Note that relative and absolute variations will be added.
\end{itemize}

\subsection{Energy range for fits}

\begin{itemize}
\item
\tt MIN\_ENERGY \textit{value}\rm\\
\tt MAX\_ENERGY \textit{value}\rm\\
These options define the range of photon beam energies $\omega$ (in MeV) for which fits are performed.
\end{itemize}

\subsection{Observable rescaling}

\begin{itemize}
\item
\tt FIX\_SCALES 1/0\rm\\
Experimental observables can be fixed to ther actual values (\tt1\rm) or scaled within their given systematic uncertainty ranges (\tt0\rm)
during the fit process. This will introduce one additional fit parameter (scaling value) for each observable.
Note that $X$ and $\sigma_X$ are considered here as different observables.

\item
\tt SCALING \textit{value}\rm\\
This is an additional weight factor that is applied to the penalty contribution imposed by the scaling factor variation.
A small \tt\textit{value}\rm\ (e.g. 0.1) will decrease this contribution and therefore
give the fitter more freedom to rescale experimental values.
\end{itemize}

\subsection{Configuration of experimental data}

\begin{itemize}
\item
\tt SG0\_FILE\bl\bl \textit{Path/Filename}\bl \textit{weight}\bl \textit{scale}\rm\\
\tt SGS\_FILE\bl\bl \textit{Path/Filename}\bl \textit{weight}\bl \textit{scale}\rm\\
\tt SGT\_FILE\bl\bl \textit{Path/Filename}\bl \textit{weight}\bl \textit{scale}\rm\\
\tt SGP\_FILE\bl\bl \textit{Path/Filename}\bl \textit{weight}\bl \textit{scale}\rm\\
\tt SGE\_FILE\bl\bl \textit{Path/Filename}\bl \textit{weight}\bl \textit{scale}\rm\\
\tt SGF\_FILE\bl\bl \textit{Path/Filename}\bl \textit{weight}\bl \textit{scale}\rm\\
\tt SGG\_FILE\bl\bl \textit{Path/Filename}\bl \textit{weight}\bl \textit{scale}\rm\\
\tt SGH\_FILE\bl\bl \textit{Path/Filename}\bl \textit{weight}\bl \textit{scale}\rm\\
\tt SGCX\_FILE\bl   \textit{Path/Filename}\bl \textit{weight}\bl \textit{scale}\rm\\
\tt SGCZ\_FILE\bl   \textit{Path/Filename}\bl \textit{weight}\bl \textit{scale}\rm\\
\tt SGOX\_FILE\bl   \textit{Path/Filename}\bl \textit{weight}\bl \textit{scale}\rm\\
\tt SGOZ\_FILE\bl   \textit{Path/Filename}\bl \textit{weight}\bl \textit{scale}\rm\\
Defines the experimental datasets for cross sections (unpolarised $\sigma_0$ and polarised $\sigma_X$). The additional
parameter \tt\textit{weight}\rm\ is applied to any $\chi^2$ contributions from the corresponding dataset. The additional
parameter \tt\textit{scale}\rm\ defines an overall scaling factor to the corresponding data and errors, i.e.
a value of $\mbox{\tt\textit{scale}\rm} = 1.05$ will increase the data by 5\%.
Multiple lines for observables with more than one dataset are possible (with individual values for
\tt\textit{weight}\rm\ and \tt\textit{scale}\rm).

\item
\tt S\_FILE\bl\bl \textit{Path/Filename}\bl \textit{weight}\bl \textit{scale}\rm\\
\tt T\_FILE\bl\bl \textit{Path/Filename}\bl \textit{weight}\bl \textit{scale}\rm\\
\tt P\_FILE\bl\bl \textit{Path/Filename}\bl \textit{weight}\bl \textit{scale}\rm\\
\tt E\_FILE\bl\bl \textit{Path/Filename}\bl \textit{weight}\bl \textit{scale}\rm\\
\tt F\_FILE\bl\bl \textit{Path/Filename}\bl \textit{weight}\bl \textit{scale}\rm\\
\tt G\_FILE\bl\bl \textit{Path/Filename}\bl \textit{weight}\bl \textit{scale}\rm\\
\tt H\_FILE\bl\bl \textit{Path/Filename}\bl \textit{weight}\bl \textit{scale}\rm\\
\tt CX\_FILE\bl   \textit{Path/Filename}\bl \textit{weight}\bl \textit{scale}\rm\\
\tt CZ\_FILE\bl   \textit{Path/Filename}\bl \textit{weight}\bl \textit{scale}\rm\\
\tt OX\_FILE\bl   \textit{Path/Filename}\bl \textit{weight}\bl \textit{scale}\rm\\
\tt OZ\_FILE\bl   \textit{Path/Filename}\bl \textit{weight}\bl \textit{scale}\rm\\
Defines the experimental datasets for asymmetry observables. The additional
parameter \tt\textit{weight}\rm\ is applied to any $\chi^2$ contributions from the corresponding dataset. The additional
parameter \tt\textit{scale}\rm\ defines an overall scaling factor to the corresponding data and errors, i.e.
a value of $\mbox{\tt\textit{scale}\rm} = 0.95$ will decrease the data by 5\%.
Multiple lines for observables with more than one dataset are possible (with individual values for
\tt\textit{weight}\rm\ and \tt\textit{scale}\rm).

\item
\tt USE\_PRELIMINARY 0/1\rm\\
Defines whether datasets tagged as `preliminary' in their respective files should be used (\tt1\rm) or not (\tt0\rm).
\end{itemize}

\subsection{Configuration of model values}

\begin{itemize}
\item
\tt MODEL\_PATH \textit{Path}\rm\\
Defines the directory containing theoretical values for multipoles. The given directory must hold text files
following the naming scheme \tt E\textit{l}p.txt\rm,
\tt E\textit{l}m.txt\rm\ 
for electric multipoles, and
\tt M\textit{l}p.txt\rm,
\tt M\textit{l}m.txt\rm\ 
for magnetic multipoles, with angular momentum \tt\textit{l}\rm, and parity 
$+$ (\tt p\rm) or $-$ (\tt m\rm).

\end{itemize}


\subsection{Special options for $\pi^0$ threshold fits}

\begin{itemize}
\item
\tt ONLY\_CROSS\_S 1/0\rm\\
\tt ONLY\_CROSS\_F 1/0\rm\\
These options restrict the data used for fitting to $\lbrace\sigma_0,\sigma_\Sigma, \Sigma\rbrace$ or
$\lbrace\sigma_0,\sigma_F, F\rbrace$. Any other observables are ignored.
These options are mutually exclusive. Also \tt FIX\_E\textit{n}P\_PHASE\rm\ must be set to \tt0\rm\
for all $s$ and $p$ wave multipoles.

\tt FIX\_RE\_E0P 1/0\rm\\
The real part of the $E_{0+}$ multipole can either be fixed (\tt1\rm) to its model value or be used as a fit parameter (\tt0\rm).
In order to use this option, both \tt FIX\_E0P\rm\ and \tt FIX\_E0P\_PHASE\rm\ must be set to \tt0\rm.

\tt FIX\_IM\_E0P\rm\\
The imaginary part of the $E_{0+}$ multipole can either be fixed (\tt1\rm) or be used as a fit parameter (\tt0\rm). If $\Im E_{0+}$
is fixed, a parametrisation according to $\Im E_{0+} = \beta \cdot \frac{q_{\pi^+}}{m_{\pi^+}}$ is used.
Note that \tt FIX\_IM\_E0P\rm\ can only be used in combination with \tt ONLY\_CROSS\_S\rm\
or \tt ONLY\_CROSS\_F\rm. Also 
both \tt FIX\_E0P\rm\ and \tt FIX\_E0P\_PHASE\rm\ must be set to \tt0\rm.

\item
\tt BETA \textit{value}\rm\\
This option sets the value for $\beta$ used in the parametrisation of $\Im E_{0+}$. Typical values are
$\beta = 3.43$ (with isospin symmetry) or $\beta = 3.35$ (with isospin breaking).

\item
\tt D\_WAVES \textit{n}\rm\\
Selects which contributions for $d$ waves is used (if supported by the current model). Possible options are
\vspace{-0.5em}\begin{enumerate}
\item[1\:\:] Full model calculation
\item[2\:\:] Born terms only
\item[3\:\:] Born terms and $\rho/\omega$ exchange
\end{enumerate}
\vspace{-0.5em}Right now, only MAID and DMT provide these three different $d$ wave contributions.

\item
\tt SGT\_ENERGIES 1/0\rm\\
With this option enabled (\tt1\rm) SE fits are performed at energies given by $\sigma_T$ data instead of $\sigma_0$.
\end{itemize}

\section{Data formats}

The following sectiosn describe the different text data formats used by PWA for experimental observable data and model multipole values. 

\subsection{Experimental observable data}

Experimental data is organised in separate files for each observable (asymmetries $X$ and polarised cross sections $\sigma_X$
are considered here as different observables) \textbf{and} for each individual measurement. A file for such an individual experiment
however can contain data for more than one energy.
Such an entry for an energy bin consists of general information on the current data as well as of the actual observable data
for different polar angle positions. A typical file looks like the following:\\

\tt
E =  683.50 MeV, E\_lo =  667.24 MeV, E\_hi =  699.76 MeV\\
Systematic = 0.0248, Preliminary = 0, CBELSA\_2014\_Hartmann\_PRL113-062001\\
163.335\bl\bl\bl -0.2596\bl\bl\bl 0.0976\bl\bl\bl 0.0080\\
151.045\bl\bl\bl -0.3346\bl\bl\bl 0.0543\bl\bl\bl 0.0086\\
142.373\bl\bl\bl -0.4980\bl\bl\bl 0.0408\bl\bl\bl 0.0129\\
135.072\bl\bl\bl -0.6306\bl\bl\bl 0.0347\bl\bl\bl 0.0154\\
128.682\bl\bl\bl -0.6693\bl\bl\bl 0.0316\bl\bl\bl 0.0168\\
122.820\bl\bl\bl -0.7317\bl\bl\bl 0.0286\bl\bl\bl 0.0170\\
117.258\bl\bl\bl -0.8295\bl\bl\bl 0.0338\bl\bl\bl 0.0261\\
112.024\bl\bl\bl -0.7583\bl\bl\bl 0.0291\bl\bl\bl 0.0170\\
106.978\bl\bl\bl -0.7793\bl\bl\bl 0.0267\bl\bl\bl 0.0180\\
102.005\bl\bl\bl -0.7629\bl\bl\bl 0.0263\bl\bl\bl 0.0189\\
\phantom{1}97.181\bl\bl\bl -0.7808\bl\bl\bl 0.0249\bl\bl\bl 0.0179\\
\phantom{1}92.407\bl\bl\bl -0.7951\bl\bl\bl 0.0247\bl\bl\bl 0.0193\\
\phantom{1}87.593\bl\bl\bl -0.7840\bl\bl\bl 0.0251\bl\bl\bl 0.0190\\
\phantom{1}82.819\bl\bl\bl -0.7738\bl\bl\bl 0.0260\bl\bl\bl 0.0179\\
\phantom{1}77.995\bl\bl\bl -0.7537\bl\bl\bl 0.0269\bl\bl\bl 0.0188\\
\phantom{1}73.022\bl\bl\bl -0.7071\bl\bl\bl 0.0278\bl\bl\bl 0.0162\\
\phantom{1}67.976\bl\bl\bl -0.6379\bl\bl\bl 0.0285\bl\bl\bl 0.0140\\
\phantom{1}62.742\bl\bl\bl -0.4937\bl\bl\bl 0.0408\bl\bl\bl 0.0125\\
----------------------------------------------------------------------\\
E =  715.61 MeV, E\_lo =  699.76 MeV, E\_hi =  731.45 MeV\\
Systematic = 0.0249, Preliminary = 0, CBELSA\_2014\_Hartmann\_PRL113-062001\\
163.335\bl\bl\bl -0.1684\bl\bl\bl 0.1025\bl\bl\bl 0.0038\\
151.045\bl\bl\bl -0.1793\bl\bl\bl 0.0599\bl\bl\bl 0.0053\\
142.373\bl\bl\bl -0.2888\bl\bl\bl 0.0447\bl\bl\bl 0.0078\\
135.072\bl\bl\bl -0.3406\bl\bl\bl 0.0414\bl\bl\bl 0.0081\\
128.682\bl\bl\bl -0.5064\bl\bl\bl 0.0369\bl\bl\bl 0.0130\\
122.820\bl\bl\bl -0.5747\bl\bl\bl 0.0335\bl\bl\bl 0.0136\\
117.258\bl\bl\bl -0.6330\bl\bl\bl 0.0405\bl\bl\bl 0.0199\\
112.024\bl\bl\bl -0.6199\bl\bl\bl 0.0360\bl\bl\bl 0.0147\\
106.978\bl\bl\bl -0.6730\bl\bl\bl 0.0335\bl\bl\bl 0.0148\\
102.005\bl\bl\bl -0.6214\bl\bl\bl 0.0312\bl\bl\bl 0.0149\\
\phantom{1}97.181\bl\bl\bl -0.6742\bl\bl\bl 0.0300\bl\bl\bl 0.0156\\
\phantom{1}92.407\bl\bl\bl -0.6291\bl\bl\bl 0.0289\bl\bl\bl 0.0149\\
\phantom{1}87.593\bl\bl\bl -0.6920\bl\bl\bl 0.0289\bl\bl\bl 0.0163\\
\phantom{1}82.819\bl\bl\bl -0.6456\bl\bl\bl 0.0301\bl\bl\bl 0.0152\\
\phantom{1}77.995\bl\bl\bl -0.5926\bl\bl\bl 0.0305\bl\bl\bl 0.0143\\
\phantom{1}73.022\bl\bl\bl -0.5031\bl\bl\bl 0.0321\bl\bl\bl 0.0117\\
\phantom{1}67.976\bl\bl\bl -0.4865\bl\bl\bl 0.0342\bl\bl\bl 0.0113\\
\phantom{1}62.742\bl\bl\bl -0.3185\bl\bl\bl 0.0396\bl\bl\bl 0.0073\\
\phantom{1}57.180\bl\bl\bl -0.2622\bl\bl\bl 0.0765\bl\bl\bl 0.0084\\
----------------------------------------------------------------------\\
\rm

The first header line carries energy information for the current dataset: The central (\tt E\rm)
photon beam energy in lab frame and the lower (\tt E\_lo\rm)
and upper (\tt E\_hi\rm) beam energy bounds of the current bin. Energies must be in MeV and the units
must be present in the header lines.
The second line gives a value for the relative systematic uncertainty (\tt Systematic\rm)
which should be an average number valid for all polar angle positions within this energy bin. This systematic uncertainty
is used for the observable rescaling option (see \tt FIX\_SCALES\rm\ configuration option). A value of
\tt Systematic = 0.05\rm\ would correspond to 5\% systematic uncertainty. 
The \tt Preliminary\rm\ field indicates whether this dataset is preliminary and/or unpublished and may not be used
for publications yet. Any data with \tt Preliminary = 1\rm\ will be ignored in fits if the configuration
option \tt USE\_PRELIMINARY\rm\ is not enabled.
The final entry in the second header line is a comment that can carry references or other identification
data for the current dataset. This commment may be up to 255 characters long and must not contain any spaces 
(use `\tt \_\rm'\ instead).

The following lines contain the actual observable data. Each line can hold up to four values which are\\
\tt\textit{theta angle\bl\bl\bl\bl observable value\bl\bl\bl\bl  statistical error\bl\bl\bl\bl}
[\textit{systematic error}]\rm\\
Polar angle positions have to be given in degrees, observable values\footnote{Asymmetry values must be within a range
of $[-1,+1]$.} for cross sections in $\mu$b/sr. Statistical and systematic
errors must be absolute values, e.g. in $\mu$b/sr for cross sections.
Systematic errors for each datapoint are optional and individual systematic uncertainties at each polar angle position
are not used by \PWA during the fit procedure.
Each energy bin entry must be terminated by a sparator line like
`\tt------------------------------\rm' (do not use blank lines to separate entries).
Such a separator however may not be present at the very first line of an observable file, but must be on the end of the file
(terminating the last entry).
Polar angle positions within an energy bin may be in increasing or decreasing order (or no order at all).
Multiple energy bins within an observable file are possible, also here the ordering in energy is not important.
Experimental data files reside in the folders \tt data/ppi0\rm, \tt data/npip\rm, 
\tt data/peta\rm, ... (depending on the reaction $p \pi^0$, $n\pi^+$, $p\eta$, ...) of your \PWA installation.
You can add or create additional files in any folder, as data files will be referenced through the 
\tt PWA.cfg\rm\ file for your fit settings.

\subsection{Model multipoles}

\PWA needs model calculations for electromagnetic multipoles for use as start parameters and (if applicable) in penalty calculations.
These multipole files must reside in a directory \tt model\rm\ in your \PWA work directory, where
\tt model\rm\ may be a link to one of the existing model directories in the model repository of the \PWA package.
Typical multipole files look like the following:\\

\tt
$\bl$\bl W \bl\bl\bl\bl\bl\bl\bl\bl M1-(p pi0)\\
(MeV)\bl\bl\bl\bl\bl\bl Re\bl\bl\bl\bl\bl\bl\bl\bl\bl\bl Im\\
1074.0\bl\bl\bl -0.389350\bl\bl\bl-0.038572\\
1075.0\bl\bl\bl -0.427039\bl\bl\bl-0.042104\\
1076.0\bl\bl\bl -0.464727\bl\bl\bl-0.045635\\
1077.0\bl\bl\bl -0.502415\bl\bl\bl-0.049167\\
1078.0\bl\bl\bl -0.540103\bl\bl\bl-0.052699\\
1079.0\bl\bl\bl -0.577792\bl\bl\bl-0.056231\\
1080.0\bl\bl\bl -0.615480\bl\bl\bl-0.059763\\
1081.0\bl\bl\bl -0.653168\bl\bl\bl-0.063294\\
1082.0\bl\bl\bl -0.690856\bl\bl\bl-0.066826\\
1083.0\bl\bl\bl -0.728545\bl\bl\bl-0.070358\\
1084.0\bl\bl\bl -0.766233\bl\bl\bl-0.073890\\
\rm

\PWA ignores the first two lines of each multipole file, so these lines can carry column headers or further descriptions.
Starting with the third line the actual multipole information has to be present in the form\\
\tt\textit{center-of-mass energy\bl\bl\bl\bl real part\bl\bl\bl\bl imaginary part}\rm\\
where the center-of-mass energy $W$ is related to the lab photon beam energy $\omega$ according to
\begin{displaymath}
 W^2 = 2M\omega + M^2
\end{displaymath}
with the mass $M$ of the target nucleon. Real and imaginary parts of the multipole must be given in units of
$10^{-3}/m_{\pi^+}$. If any model provides multipoles in units of $10^{-3}\:\mathrm{fm}$ a conversion factor of
\begin{displaymath}
 \frac{m_{\pi^+}}{\alpha_\mathrm{em}} =\frac{139.5702\:\mathrm{MeV}}{197.3270\:\mathrm{MeV\cdot fm}}
\approx \sqrt\frac{1}{2}\:\mathrm{fm^{-1}}
\end{displaymath}
has to be applied before using these model calculations with \tt PWA\rm.
Multipole files have to follow the naming scheme
\tt E\textit{l}p.txt\rm,
\tt E\textit{l}m.txt\rm\ 
for electric multipoles, and
\tt M\textit{l}p.txt\rm,
\tt M\textit{l}m.txt\rm\ 
for magnetic multipoles, with angular momentum \tt\textit{l}\rm, and parity 
$+$ (\tt p\rm) or $-$ (\tt m\rm).
\PWA supports multipoles up to order $L = 9$.
The energy steps in multipole files can be arbitrary, however a rather fine step size ($\sim 1\:\mathrm{MeV}$)
is recommended in order to have reasonably precise multipole information for any SE fit energy available
(up to now, no interpolation on model values is performed).
For $d$ wave multipoles additional model values may be provided, carrying only Born term contributions or Bern terms as well as 
$\rho/\omega$ exchange mechanisms (see configuration option \tt D\_WAVES\rm).
These files must be named
\tt E\textit{2}p\_Born.txt\rm,
\tt E\textit{2}m\_Born.txt\rm,
\tt M\textit{2}p\_Born.txt\rm,
\tt M\textit{2}m\_Born.txt\rm,
and
\tt E\textit{2}p\_BornRhoOmega.txt\rm,
\tt E\textit{2}m\_BornRhoOmega.txt\rm,
\tt M\textit{2}p\_BornRhoOmega.txt\rm,
\tt M\textit{2}m\_BornRhoOmega.txt\rm,
respectively.
All filenames are case-sensitive.

Currently, for $p \pi^0$ the following models are provided: \vspace{-0.25em}\\
\begin{tabular}{ll}
\hspace{-0.5em}\tt model/ppi0/BG2011-01\rm & Bonn-Gatchina PWA (solution 2011-01) \cite{Model_BnGa} \vspace{-0.5em}\\
\hspace{-0.5em}\tt model/ppi0/BG2011-02\rm & Bonn-Gatchina PWA (solution 2011-02) \cite{Model_BnGa} \vspace{-0.5em}\\
\hspace{-0.5em}\tt model/ppi0/DMT\rm & DMT model \cite{Model_DMT} \vspace{-0.5em}\\
\hspace{-0.5em}\tt model/ppi0/MAID\rm & MAID 2007 \cite{Model_MAID} \vspace{-0.5em}\\
\hspace{-0.5em}\tt model/ppi0/SAID\rm & SAID CM12 \cite{Model_SAID} \vspace{-0.5em}\\
\hspace{-0.5em}\tt model/ppi0/ThrFit\rm & Empirical fit to $\pi^0$ threshold data \cite{PRL_pi0thres} \vspace{-0.5em}\\
\hspace{-0.5em}\tt model/ppi0/ThrChPT4\rm & $\chi$PT (4$^\mathrm{th}$ order) \cite{Model_ChPT4} \vspace{-0.5em}\\
\hspace{-0.5em}\tt model/ppi0/ThrHBChPT4\rm & Heavy-baryon $\chi$PT (4$^\mathrm{th}$ order) \cite{Model_HBChPT4} \vspace{-0.5em}\\
\hspace{-0.5em}\tt model/ppi0/ThrHDT97\rm & Hanstein, Drechsel, Tiator solution \cite{Model_HDT97} \vspace{-0.5em}\\
\hspace{-0.5em}\tt model/ppi0/ThrLG\rm & Lutz, Gasparyan calculation \cite{Model_LG} \vspace{-0.1em}\\
\end{tabular}

Any models prefixed with \tt Thr\rm\ are only supposed to be used within the $\pi$ threshold region up to
$\omega\sim 180\mathrm{MeV}$.

For $n \pi^+$ the following models are provided: \vspace{-0.25em}\\
\begin{tabular}{ll}
\hspace{-0.5em}\tt model/ppi0/BG2011-01\rm & Bonn-Gatchina PWA (solution 2011-01) \cite{Model_BnGa} \vspace{-0.5em}\\
\hspace{-0.5em}\tt model/ppi0/BG2011-02\rm & Bonn-Gatchina PWA (solution 2011-02) \cite{Model_BnGa} \vspace{-0.5em}\\
\hspace{-0.5em}\tt model/ppi0/DMT\rm & DMT model \cite{Model_DMT} \vspace{-0.5em}\\
\hspace{-0.5em}\tt model/ppi0/MAID\rm & MAID 2007 \cite{Model_MAID} \vspace{-0.5em}\\
\hspace{-0.5em}\tt model/ppi0/SAID\rm & SAID CM12 \cite{Model_SAID} \vspace{-0.1em}\\
\end{tabular}

For $p \eta$ the following models are provided: \vspace{-0.25em}\\
\begin{tabular}{ll}
\hspace{-0.5em}\tt model/ppi0/BG2011-01\rm & Bonn-Gatchina PWA (solution 2011-01) \cite{Model_BnGa} \vspace{-0.5em}\\
\hspace{-0.5em}\tt model/ppi0/BG2011-02\rm & Bonn-Gatchina PWA (solution 2011-02) \cite{Model_BnGa} \vspace{-0.5em}\\
\hspace{-0.5em}\tt model/ppi0/DMT\rm & DMT model \cite{Model_DMT} \vspace{-0.5em}\\
\hspace{-0.5em}\tt model/ppi0/MAID\rm & $\eta$MAID 2000 \cite{Model_MAID} \vspace{-0.5em}\\
\hspace{-0.5em}\tt model/ppi0/SAID\rm & SAID \cite{Model_SAID} \vspace{-0.1em}\\
\end{tabular}


\section{Macros}

\subsection{Macro functions in  \tt macros/Extract.cpp\rm}

\begin{itemize}
\item
\tt Extract(Char\_t* REACT, Int\_t L\_MAX, Int\_t SOLUTIONS=1, Double\_t MASS\_INITIAL=938.2720)\rm

Converts \tt .root\rm\ files from \PWA output to plain text files. \tt Extract()\rm\ will
create text files for all multipoles up to multipole order \tt L\_MAX\rm\ from fit results in directories
\tt plots.0\rm\ up to \tt plots.$\lbrace$\textit{SOLUTIONS}-1$\rbrace$\rm.
\tt REACT\rm\ is an (arbitrary) reaction identifier that will be written in the header lines of output files,
e.g. \tt ''p pi0''\rm. \tt MASS\_INITIAL\rm\ (in MeV) is used for calculating the center-of-mass energy $W$. This parameter 
is optional, if it is omitted the proton mass is assumed.
Output files will be named
\tt E\textit{l}p.txt\rm,
\tt E\textit{l}m.txt\rm\ 
for electric multipoles, and
\tt M\textit{l}p.txt\rm,
\tt M\textit{l}m.txt\rm\ 
for magnetic multipoles, with angular momentum \tt\textit{l}\rm, and parity
$+$ (\tt p\rm) or $-$ (\tt m\rm). These files
will be created in the corresponding 
\tt plots.\textit{n}\rm\ directories and have the following format:\\

\tt
$\bl$\bl\bl W \bl\bl\bl\bl\bl\bl\bl\bl\bl\bl\bl\bl\bl\bl\bl E0+(p pi0)\\
\bl(MeV)\bl\bl\bl\bl\bl\bl Re\bl\bl\bl\bl\bl\bl DRe\bl\bl\bl\bl\bl\bl Im\bl\bl\bl\bl\bl\bl DIm\\
1074.018\bl\bl\bl -0.837\bl\bl\bl 0.107\bl\bl\bl-0.028\bl\bl\bl 0.195\\
1075.131\bl\bl\bl -0.793\bl\bl\bl 0.063\bl\bl\bl-0.253\bl\bl\bl 0.325\\
1076.242\bl\bl\bl -1.670\bl\bl\bl 0.801\bl\bl\bl-0.905\bl\bl\bl 1.106\\
1077.351\bl\bl\bl -2.188\bl\bl\bl 0.243\bl\bl\bl-0.112\bl\bl\bl 4.192\\
1078.460\bl\bl\bl -0.541\bl\bl\bl 0.027\bl\bl\bl\bl 0.408\bl\bl\bl 0.243\\
\rm

Each line consists of the center-of-mass energy $W$ of an SE fit as well as values and errors for real and imaginary
parts obtained from the fit. \textbf{Note:} After execution of \tt Extract()\rm\ it is recommended to restart
\tt ROOT\rm\ to avoid problems when drawing \tt TCanvas\rm\ objects. \tt Extract()\rm\ must be executed from your
work directory (having subdirectories \tt plots.\textit{n}\rm).

\item
\tt Multipole(Char\_t* Mlp, Int\_t SolLo=0, Int\_t SolHi=0, Bool\_t W=true,\\
\phantom{Multipole(}Bool\_t SAVE=false, Double\_t Lo=0.0, Double\_t Hi=0.0,\\
\phantom{Multipole(}Double\_t MASS\_INITIAL=938.2720, Int\_t D\_WAVES=MODEL)\rm

Plots the given multipole (real and imaginary part) \tt Mlp\rm\ for solutions between \tt SolLo\rm\ and \tt SolHi\rm.
\tt Mlp\rm\ must be given in the form \tt ''E\textit{l}p''\rm,
\tt ''E\textit{l}m''\rm\ 
for electric multipoles, and
\tt ''M\textit{l}p''\rm,
\tt ''M\textit{l}m''\rm\ 
for magnetic multipoles, with angular momentum \tt\textit{l}\rm, and parity
$+$ (\tt p\rm) or $-$ (\tt m\rm). The plot will contain SE fit results as well as model predictions from the current
values in the directory \tt model\rm.
if \tt SAVE\rm\ is true, a PDF file of the plot is created. \tt Lo\rm\ and \tt Hi\rm\ denote the plot range on the $y$-axis, 
if ommitted or equal 0, the range is automatically adjusted.
\tt MASS\_INITIAL\rm\ (in MeV) is used for calculating the center-of-mass energy $W$. This parameter 
is optional, if it is omitted the proton mass is assumed.
\tt D\_WAVES\rm\ indicates, which $d$ wave contributions are used for plotting model multipoles (if supported by the current model).
\textbf{Note:} For successful operation of \tt Multipole()\rm\ it is necessary to have produced
fitted multipole text files with \tt Extract()\rm\ beforehand once. \tt Multipole()\rm\ must be executed from your
work directory (having subdirectories \tt plots.\textit{n}\rm\ and \tt model\rm).

\item
\tt Magnitude(Char\_t* Mlp, Int\_t SolLo=0, Int\_t SolHi=0, Bool\_t W=true,\\
\phantom{Magnitude(}Bool\_t SAVE=false, Double\_t Lo=0.0, Double\_t Hi=0.0,\\
\phantom{Magnitude(}Double\_t MASS\_INITIAL=938.2720, Int\_t D\_WAVES=MODEL)\rm

Plots the magnitude of the given multipole \tt Mlp\rm\ for solutions between \tt SolLo\rm\ and \tt SolHi\rm.
\tt Mlp\rm\ must be given in the form \tt ''E\textit{l}p''\rm,
\tt ''E\textit{l}m''\rm\ 
for electric multipoles, and
\tt ''M\textit{l}p''\rm,
\tt ''M\textit{l}m''\rm\ 
for magnetic multipoles, with angular momentum \tt\textit{l}\rm, and parity
$+$ (\tt p\rm) or $-$ (\tt m\rm). The plot will contain SE fit results as well as model predictions from the current
values in the directory \tt model\rm.
if \tt SAVE\rm\ is true, a PDF file of the plot is created. \tt Lo\rm\ and \tt Hi\rm\ denote the plot range on the $y$-axis, 
if ommitted or equal 0, the range is automatically adjusted.
\tt MASS\_INITIAL\rm\ (in MeV) is used for calculating the center-of-mass energy $W$. This parameter 
is optional, if it is omitted the proton mass is assumed.
\tt D\_WAVES\rm\ indicates, which $d$ wave contributions are used for plotting model multipoles (if supported by the current model).
\textbf{Note:} For successful operation of \tt Magnitude()\rm\ it is necessary to have produced
fitted multipole text files with \tt Extract()\rm\ beforehand once. \tt Magnitude()\rm\ must be executed from your
work directory (having subdirectories \tt plots.\textit{n}\rm\ and \tt model\rm).

\item
\tt Phase(Char\_t* Mlp, Int\_t SolLo=0, Int\_t SolHi=0, Bool\_t W=true,\\
\phantom{Phase(}Bool\_t SAVE=false, Double\_t Lo=0.0, Double\_t Hi=0.0,\\
\phantom{Phase(}Double\_t MASS\_INITIAL=938.2720, Int\_t D\_WAVES=MODEL)\rm

Plots the phase (in radians) of the given multipole \tt Mlp\rm\ for solutions between \tt SolLo\rm\ and \tt SolHi\rm.
\tt Mlp\rm\ must be given in the form \tt ''E\textit{l}p''\rm,
\tt ''E\textit{l}m''\rm\ 
for electric multipoles, and
\tt ''M\textit{l}p''\rm,
\tt ''M\textit{l}m''\rm\ 
for magnetic multipoles, with angular momentum \tt\textit{l}\rm, and parity
$+$ (\tt p\rm) or $-$ (\tt m\rm). The plot will contain SE fit results as well as model predictions from the current
values in the directory \tt model\rm.
if \tt SAVE\rm\ is true, a PDF file of the plot is created. \tt Lo\rm\ and \tt Hi\rm\ denote the plot range on the $y$-axis, 
if ommitted or equal 0, the range is automatically adjusted.
\tt MASS\_INITIAL\rm\ (in MeV) is used for calculating the center-of-mass energy $W$. This parameter 
is optional, if it is omitted the proton mass is assumed.
\tt D\_WAVES\rm\ indicates, which $d$ wave contributions are used for plotting model multipoles (if supported by the current model).
\textbf{Note:} For successful operation of \tt Phase()\rm\ it is necessary to have produced
fitted multipole text files with \tt Extract()\rm\ beforehand once. \tt Phase()\rm\ must be executed from your
work directory (having subdirectories \tt plots.\textit{n}\rm\ and \tt model\rm).

\item
\tt Model(Char\_t* Mlp, Bool\_t W=true, Bool\_t SAVE=false, Double\_t Lo=0.0, Double\_t Hi=0.0,\\
\phantom{Model(}Double\_t MASS\_INITIAL=938.2720, Int\_t D\_WAVES=MODEL)\rm

Plots model predictions for the given multipole (real and imaginary part) \tt Mlp\rm.
\tt Mlp\rm\ must be given in the form \tt ''E\textit{l}p''\rm,
\tt ''E\textit{l}m''\rm\ 
for electric multipoles, and
\tt ''M\textit{l}p''\rm,
\tt ''M\textit{l}m''\rm\ 
for magnetic multipoles, with angular momentum \tt\textit{l}\rm, and parity
$+$ (\tt p\rm) or $-$ (\tt m\rm). The plot will contain only model predictions from the current
values in the directory \tt model\rm.
if \tt SAVE\rm\ is true, a PDF file of the plot is created. \tt Lo\rm\ and \tt Hi\rm\ denote the plot range on the $y$-axis, 
if ommitted or equal 0, the range is automatically adjusted.
\tt MASS\_INITIAL\rm\ (in MeV) is used for calculating the center-of-mass energy $W$. This parameter 
is optional, if it is omitted the proton mass is assumed.
\tt D\_WAVES\rm\ indicates, which $d$ wave contributions are used for plotting model multipoles (if supported by the current model).
\textbf{Note:}  \tt Model()\rm\ must be executed from your
work directory (having a subdirectory \tt model\rm).


\item
\tt Chi2(Int\_t SOLUTION=0, Double\_t MASS\_INITIAL=938.2720)\rm

Draws the $\chi^2$ values for the given \tt SOLUTION\rm\ depending on the center-of-mass energy $W$.
\tt MASS\_INITIAL\rm\ (in MeV) is used for calculating $W$. This parameter 
is optional, if it is omitted the proton mass is assumed.
\tt Chi2()\rm\ must be executed from your
work directory (having subdirectories \tt plots.\textit{n}\rm).

\item
\tt Penalty(Int\_t SOLUTION=0, Double\_t MASS\_INITIAL=938.2720)\rm

Draws the penalty contributions for the given \tt SOLUTION\rm\ depending on the center-of-mass energy $W$.
\tt MASS\_INITIAL\rm\ (in MeV) is used for calculating $W$. This parameter 
is optional, if it is omitted the proton mass is assumed.
\tt Penalty()\rm\ must be executed from your
work directory (having subdirectories \tt plots.\textit{n}\rm).
\end{itemize}



\begin{thebibliography}{00}

\bibliographystyle{unsrt}

\bibitem{PRL_pi0thres}
D. Hornidge et al.,
Phys. Rev. Lett. \textbf{111}, 062004 (2013)

\bibitem{Model_BnGa}

\bibitem{Model_DMT}

\bibitem{Model_MAID}

\bibitem{Model_SAID}

\bibitem{Model_ChPT4}

\bibitem{Model_HBChPT4}

\bibitem{Model_HDT97}

\bibitem{Model_LG}

\end{thebibliography}

\end{document}

