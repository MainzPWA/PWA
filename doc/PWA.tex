\documentclass[a4paper,10pt]{article}
\usepackage{afterpage}
\usepackage{amssymb}
\usepackage{amsmath}
\usepackage{color}
\usepackage{enumerate}
\usepackage[latin1]{inputenc}
%\usepackage{times}
%\usepackage{mathptmx}
%\usepackage{t1enc}
%\usepackage{german}
\usepackage{graphicx}
\usepackage{latexsym}
\usepackage{mathrsfs}
%\usepackage{oldgerm}
\usepackage{pifont}
\usepackage{psfrag}
\usepackage{rotating}
%\usepackage{showkeys}
\usepackage[german]{varioref}
\usepackage{wrapfig}
\usepackage{wasysym}
%\usepackage{txfonts}
\usepackage{ulem}
\usepackage[T1]{fontenc} 
%\usepackage[helvet]{sfmath}
%\usepackage{helvet}
\usepackage{eurosym}
\usepackage{geometry}
\geometry{left=20mm,right=20mm, top=2cm, bottom=2cm} 
\parindent=0cm

\def\PWA{\ttfamily PWA\rmfamily\ }
\def\d{\mathrm{d}}
\def\cc{^*}
\def\Re{\mathrm{Re}}
\def\Im{\mathrm{Im}}
\def\enn{\mbox{\ttfamily\textit{n}\rmfamily}}
\def\bl{\phantom{0}}
\def\tt{\ttfamily}
\def\rm{\rmfamily}
\renewcommand{\arraystretch}{1.5}

% Title Page
\title{Using the Mainz Multipole Fitter \PWA}
\author{Sven Schumann}

\begin{document}
\maketitle

\begin{abstract}
\hspace{-1.5em}This document describes details and usage of the Mainz \PWA tool used to determine electromagnetic multipoles
for pseudoscalar meson photoproduction from experimental results for cross sections and various polarisation
observables. \PWA performs constrained single energy (SE) fits based on a $\chi^2$ minimisation scheme with additional
penalty terms.
\end{abstract}

\section{Introduction}

\PWA is a MINUIT-based single energy (SE) multipole fitting tool, used to determine electromagnetic multipoles for pseudoscalar
meson photoproduction ($\pi$, $\eta$, $K$, ...) from experimental cross sections (polarised and unpolarised)
and/or asymmetry observables. Multipoles are extracted using a standard $\chi^2$ minimisation scheme with the
possibility to add additional constraints imposed by existing partial wave analyses and models for meson photoproduction
(MAID, SAID, BnGa, ...). These constraints are implemented using so-called penalty terms, giving an additional
contribution in the minimisation process. The maximum multipole order $L_\mathrm{max}$ which is considered to contribute to
physical observables can be adjusted up to $L_\mathrm{max} = 10$. Indvidiual multipoles can either be fitted in
real and imaginary parts, with fixed phases but fittable magnitude, or can be fixed to model predictions completely.

\section{$\chi^2$ minimisation and multipole extraction}

\PWA determines electromagnetic multipoles as fit parameters in a $\chi^2$ minimisation process with respect to
experimental photoproduction results in form of the standard photoproduction observables, which are typically
given in four groups
\begin{displaymath}
\begin{array}{ll}
 \mbox{Group $S$:} &\frac{\d\sigma}{\d\Omega} = \sigma_0, \Sigma, T, P\\
 \mbox{Group $BT$:}& E, F, G, H\\
 \mbox{Group $BR$:}& C_{x^\prime}, C_{z^\prime}, O_{x^\prime}, O_{z^\prime}\\
 \mbox{Group $TR$:}& T_{x^\prime}, T_{z^\prime}, L_{x^\prime}, L_{z^\prime}
\end{array}
\end{displaymath}
\PWA supports observable data for all groups $S$, $BT$, $BR$, and $TR$. These observables can be expressed
in form of the four CGLN amplitudes $F_i$ according to
\begin{displaymath}\label{frm_Obs}
\begin{array}{lcl}
\sigma_0 &\hspace{-0.67em}=&\hspace{-0.5em} \Re \left\lbrace |F_1|^2 + |F_2|^2 - 2\cos\theta F_1\cc F_2  + \frac{1}{2}\sin^2\theta\left( |F_3|^2 + |F_4|^2 + 2 F_2\cc  F_3    + 2 F_1\cc  F4 + 2\cos\theta F_3\cc F_4\right) \right\rbrace \cdot \rho\\
\sigma_0 \Sigma &\hspace{-0.67em}=&\hspace{-0.5em} - \frac{1}{2}\sin^2\theta\:\Re\left\lbrace|F_3|^2 + |F_4|^2 
                                               + 2\left(F_2\cc F_3 + F_1\cc F_4 
                                               + \cos\theta F_3\cc F_4\right)  
                                               \right\rbrace \cdot \rho\\
\sigma_0 T     &\hspace{-0.67em}=&\hspace{-0.5em} \sin\theta\:\Im \left\lbrace F_1\cc F_3 - F_2\cc F_4 
                                               + \cos\theta\left(F_1\cc F_4 - F_2\cc F_3\right)
                                               - \sin^2\theta F_3\cc F_4
                                               \right\rbrace \cdot \rho\\
\sigma_0 P     &\hspace{-0.67em}=&\hspace{-0.5em} \sin\theta\: \Im\left\lbrace F_2\cc F_4 - 2 F_1\cc F_2 - F_1 F_3
                                               + \cos\theta\left(F_2\cc F_3 - F_1\cc F_4\right) 
                                               + \sin^2\theta F_3\cc F_4
                                               \right\rbrace \cdot \rho\\
\sigma_0 E     &\hspace{-0.67em}=&\hspace{-0.5em} \Re\left\lbrace|F_1|^2 + |F_2|^2 - 2\cos\theta F_1\cc F_2
                                               +\sin^2\theta\left(F_2\cc F_3 + F_1\cc F_4\right)
                                               \right\rbrace \cdot \rho\\
\sigma_0 F     &\hspace{-0.67em}=&\hspace{-0.5em} \sin\theta \:\Re\left\lbrace F_1\cc F_3 - F_2\cc F_4 
                                               + \cos\theta\left(F_1\cc F_4 + F_2\cc F_3\right)
                                                 \right\rbrace \cdot \rho\\
\sigma_0 G     &\hspace{-0.67em}=&\hspace{-0.5em} \sin^2\theta \:\Im\left\lbrace F_2\cc F_3 + F_1\cc F_4\:
                                                \right\rbrace \cdot \rho\\
\sigma_0 H_{\phantom{z^\prime}} &\hspace{-0.67em}=&\hspace{-0.5em} \sin\theta \:\Im \left\lbrace 2 F_1\cc F_2 + F_1\cc F_3 
                                               - F_2\cc F_4 -\cos\theta\left( F_2\cc F_3 - F_1\cc F_4 \right)
                                                 \right\rbrace \cdot \rho\\
&\hspace{-0.67em}&\hspace{-0.5em}\phantom{\Re\left\lbrace 2F_1\cc F_2 - \cos\theta\left(|F_1|^2  + |F_2|^2\right)
           + \sin^2\theta\left(F_1\cc F_3 + F_2\cc F_4 + F_3\cc F_4\right)
           + \frac{1}{2}\cos\theta\sin^2\theta\left(|F_3|^2 + |F_4|^2\right) \right\rbrace \cdot \rho}\vspace{-1.5em}\\
\end{array}
\end{displaymath}
\begin{displaymath}\label{frm_Obs2}
\begin{array}{lcl}
\sigma_0 O_{x^\prime} &\hspace{-0.67em}=&\hspace{-0.5em} -\sin\theta \:\Im \left\lbrace F_1\cc F_4 - F_2\cc F_3
                                                      + \cos\theta\left(F_1\cc F_3 - F_2\cc F_4\right)
                                                       \right\rbrace \cdot \rho\\
\sigma_0 O_{z^\prime} &\hspace{-0.67em}=&\hspace{-0.5em} -\sin^2\theta \:\Im \left\lbrace F_1\cc F_3 + F_2\cc F_4\:
                                                       \right\rbrace \cdot \rho\\
\sigma_0 C_{x^\prime} &\hspace{-0.67em}=&\hspace{-0.5em} \sin\theta \:\Re \left\lbrace |F_1|^2 - |F_2|^2 
                                                      + F_1\cc F_3 - F_2\cc F_3 
                                                      +\cos\theta\left(F_1\cc F_3 - F_2\cc F_4\right)
                                                       \right\rbrace \cdot \rho\\
\sigma_0 C_{z^\prime} &\hspace{-0.67em}=&\hspace{-0.5em} \Re \left\lbrace 2 F_1\cc F_2 
                                                      +\sin^2\theta\left(F_1\cc F_3 + F_2\cc F_4\right)
                                                      -\cos\theta\left(|F_1|^2+|F_2|^2\right)
                                                       \right\rbrace \cdot \rho\\
\sigma_0 L_{x^\prime} &\hspace{-0.67em}=&\hspace{-0.5em} -\sin\theta \:\Re\left\lbrace |F_1|^2 - |F_2|^2 - F_2\cc F_3  + F_1\cc F_4
                            + \frac{1}{2}\sin^2\theta \left( |F_4|^2 - |F_3|^2\right) 
                            + \cos\theta \left(F_1\cc F_3 - F_2\cc F_4\right) \right\rbrace \cdot \rho\\
\sigma_0 L_{z^\prime} &\hspace{-0.67em}=&\hspace{-0.5em} \Re\left\lbrace 2F_1\cc F_2 - \cos\theta\left(|F_1|^2  + |F_2|^2\right)
                             + \sin^2\theta\left(F_1\cc F_3 + F_2\cc F_4 + F_3\cc F_4\right)
                             + \frac{1}{2}\cos\theta\sin^2\theta\left(|F_3|^2 + |F_4|^2\right) \right\rbrace \cdot \rho\\
\sigma_0 T_{x^\prime} &\hspace{-0.67em}=&\hspace{-0.5em} -\sin^2\theta\:\Re \left\lbrace F_1\cc F_3 + F_2\cc F_4 + F_3\cc F_4
                            +\frac{1}{2}\cos\theta\left(|F_3|^2 + |F_4|^2\right) \right\rbrace \cdot \rho\\
\sigma_0 T_{z^\prime} &\hspace{-0.67em}=&\hspace{-0.5em} \sin\theta\:\Re \left\lbrace F_1\cc F_4 - F_2\cc F_3
                            + \cos\theta\left(F_1\cc F_3 - F_2\cc F_4\right)
                            +\frac{1}{2}\sin^2\theta\left(F_4||^ - |F_3|^2\right)\right\rbrace \cdot \rho\\
\end{array}
\end{displaymath}
with a phase space factor $\rho = \frac{q}{k}$ given by pion and photon momenta $q$ and $k$. The CGLN amplitudes
depend on the electromagnetic multipoles $E_{L\pm}$, $M_{L\pm}$ for multipole orders $L$ up to $L_\mathrm{max}$
\begin{displaymath}\label{frm_CGLN}
\begin{array}{l}
F_1 = \sum\limits_{L\geq 0}^{L_\mathrm{max}} \left\lbrace (L\cdot M_{L+} + E_{L+})P^{\prime}_{L+1} 
+ \left[(L+1)\cdot M_{L-} + E_{L-}\phantom{{}^{\prime}_{L}}\right]P^{\prime}_{L-1} \right\rbrace\\
F_2 = \sum\limits_{L\geq 1}^{L_\mathrm{max}} \left[(L+1)\cdot M_{L+} + L\cdot M_{L-} \phantom{{}^{\prime}_{L}}  \right]P^{\prime}_{L}\\
F_3 = \sum\limits_{L\geq 1}^{L_\mathrm{max}} \left[(E_{L+} - M_{L+})P^{\prime\prime}_{L+1}+ (E_{L-} + M_{L-})P^{\prime\prime}_{L-1} \right]\\
F_4 = \sum\limits_{L\geq 2}^{L_\mathrm{max}}\left( M_{L+} - E_{L+} - M_{L-} - E_{L-}\right)P^{\prime\prime}_{L}
\end{array}
\end{displaymath}
and derivatives of the Legendre polynoms $P_L$ describing the angular dependencies. Using these relations between observables
and electromagnetic multipoles, it is possible to create a $\chi^2$ function having multipoles as fit parameters and physical
observables as experimental input
\begin{displaymath}
\begin{array}{lcl}
\chi^2
&=& \sum\limits_i \left(
\frac
{\sigma_0^\mathrm{exp}(\theta_i) - \sigma_0^\mathrm{fit}(\theta_i)}
{\Delta \sigma_0^\mathrm{exp}(\theta_i)}
\right)^2\\
&+& \sum\limits_i \left(
\frac
{\Sigma^\mathrm{exp}(\theta_i) - \Sigma^\mathrm{fit}(\theta_i)}
{\Delta \Sigma^\mathrm{exp}(\theta_i)}
\right)^2
+ \sum\limits_i \left(
\frac
{\sigma_\Sigma^\mathrm{exp}(\theta_i) - \sigma_\Sigma^\mathrm{fit}(\theta_i)}
{\Delta \sigma_\Sigma^\mathrm{exp}(\theta_i)}
\right)^2\\
&+& \sum\limits_i \left(
\frac
{T^\mathrm{exp}(\theta_i) - T^\mathrm{fit}(\theta_i)}
{\Delta T^\mathrm{exp}(\theta_i)}
\right)^2
+ \sum\limits_i \left(
\frac
{\sigma_T^\mathrm{exp}(\theta_i) - \sigma_T^\mathrm{fit}(\theta_i)}
{\Delta \sigma_T^\mathrm{exp}(\theta_i)}
\right)^2\\
&+& \sum\limits_i \left(
\frac
{P^\mathrm{exp}(\theta_i) - P^\mathrm{fit}(\theta_i)}
{\Delta P^\mathrm{exp}(\theta_i)}
\right)^2
+ \sum\limits_i \left(
\frac
{\sigma_P^\mathrm{exp}(\theta_i) - \sigma_P^\mathrm{fit}(\theta_i)}
{\Delta \sigma_P^\mathrm{exp}(\theta_i)}
\right)^2\\
&+& \sum\limits_i \left(
\frac
{E^\mathrm{exp}(\theta_i) - E^\mathrm{fit}(\theta_i)}
{\Delta E^\mathrm{exp}(\theta_i)}
\right)^2
+ \sum\limits_i \left(
\frac
{\sigma_E^\mathrm{exp}(\theta_i) - \sigma_E^\mathrm{fit}(\theta_i)}
{\Delta \sigma_E^\mathrm{exp}(\theta_i)}
\right)^2\\
&+& \sum\limits_i \left(
\frac
{F^\mathrm{exp}(\theta_i) - F^\mathrm{fit}(\theta_i)}
{\Delta F^\mathrm{exp}(\theta_i)}
\right)^2
+ \sum\limits_i \left(
\frac
{\sigma_F^\mathrm{exp}(\theta_i) - \sigma_F^\mathrm{fit}(\theta_i)}
{\Delta \sigma_F^\mathrm{exp}(\theta_i)}
\right)^2\\
&+& \sum\limits_i \left(
\frac
{G^\mathrm{exp}(\theta_i) - G^\mathrm{fit}(\theta_i)}
{\Delta G^\mathrm{exp}(\theta_i)}
\right)^2
+ \sum\limits_i \left(
\frac
{\sigma_G^\mathrm{exp}(\theta_i) - \sigma_G^\mathrm{fit}(\theta_i)}
{\Delta \sigma_G^\mathrm{exp}(\theta_i)}
\right)^2\\
&+& \sum\limits_i \left(
\frac
{H^\mathrm{exp}(\theta_i) - H^\mathrm{fit}(\theta_i)}
{\Delta H^\mathrm{exp}(\theta_i)}
\right)^2
+ \sum\limits_i \left(
\frac
{\sigma_H^\mathrm{exp}(\theta_i) - \sigma_H^\mathrm{fit}(\theta_i)}
{\Delta \sigma_H^\mathrm{exp}(\theta_i)}
\right)^2
\end{array}
\end{displaymath}
where $\sigma_X^\mathrm{exp}(\theta_i)$, $X^\mathrm{exp}(\theta_i)$ describe experimental cross sections and asymmetries
at their polar angle positions $\theta_i$, while 
$\sigma_X^\mathrm{fit}(\theta_i)$, $X^\mathrm{fit}(\theta_i)$ are cross sections and asymmetries as calculated from
multipole parameters $E_{L\pm}$, $M_{L\pm}$.
Minimisation of the above $\chi^2$ function will therefore adjust electromagnetic multipole values in order for the best description
of available data.
\PWA will use asymmetries $X$ as well as polarised cross sections $\sigma_X = \sigma_0\cdot X$ as experimental input,
so there is no need to transform one type of experimental information to another beforehand.

\section{Penalty terms}

Penalty terms act as an additional contribution to the function to be minimised during the fit process, i.e. fitting with penalty
no longer mminimises the `pure' $\chi^2$ but instead
\begin{displaymath}
 f_\mathrm{FCN} = \chi^2 + Q
\end{displaymath}
where $Q$ is typically designed in a way that it gets larger if current fit parameters deviate from a given
model solution. Hence any deviation from the used model is `penalised' in the fitting. The found solution
for fit parameters is therefore a compromise between the best description of the available data (contributing to $\chi^2$)
and an agreement with a given model calculation (contributing to $Q$).
Penalty terms in the $f_\mathrm{FCN}$ function are useful to ensure that results of SE fits at different energies
fulfill constraints like a smooth energy dependence and can help to get reasonable fit results
if experimental data is too scarce (in terms of the number of observables or angular coverage) or has limited precision.
If however the penalty terms dominate the $f_\mathrm{FCN}$
function, the found solutions will mainly try to reproduce the initial model calculation. Therefore a penalty-supported fit 
is never a fully model-independent solution but will always show some bias caused by the used model calculation.
Proper adjustment of penalty and $\chi^2$ constributions is crucial in order to get meaningful fit results.
\PWA supports a variety of different penalty implementations that either evaluate model multipoles directly or with
an intermediate step of meson photoproduction amplitudes.

\subsection{Penalty mode \textit{MLP1}}

The \textit{MLP1} penalty calculation evaluates a penalty term $Q_1$ from the squared sum of all absolute deviations
between fitted 
(${E}_{L\pm}^\mathrm{fit}$, ${M}_{L\pm}^\mathrm{fit}$)
and model
(${E}_{L\pm}^\mathrm{sol}$, ${M}_{L\pm}^\mathrm{sol}$)
multipoles, according to
\begin{displaymath}
\begin{array}{lcl}
 Q_1 &=& \dfrac{q_1}{|\mathcal{M}^\mathrm{sol}|^2}\cdot \dfrac{N_\mathrm{pts}}{N_\mathrm{par}} \cdot \left[
\sum\limits_L^{L_\mathrm{max}}\left(\Re E_{L-}^\mathrm{fit} - \Re E_{L-}^\mathrm{sol}\right)^2
+
\sum\limits_L^{L_\mathrm{max}}\left(\Im E_{L-}^\mathrm{fit} - \Im E_{L-}^\mathrm{sol}\right)^2\right.\\
&&\hspace{6.73em}+
\sum\limits_L^{L_\mathrm{max}}\left(\Re E_{L+}^\mathrm{fit} - \Re E_{L+}^\mathrm{sol}\right)^2
+
\sum\limits_L^{L_\mathrm{max}}\left(\Im E_{L+}^\mathrm{fit} - \Im E_{L+}^\mathrm{sol}\right)^2\\
&&\hspace{6.73em}+
\sum\limits_L^{L_\mathrm{max}}\left(\Re M_{L-}^\mathrm{fit} - \Re M_{L-}^\mathrm{sol}\right)^2
+
\sum\limits_L^{L_\mathrm{max}}\left(\Im M_{L-}^\mathrm{fit} - \Im M_{L-}^\mathrm{sol}\right)^2\\
&&\hspace{6.73em}+\left.
\sum\limits_L^{L_\mathrm{max}}\left(\Re M_{L+}^\mathrm{fit} - \Re M_{L+}^\mathrm{sol}\right)^2
+
\sum\limits_L^{L_\mathrm{max}}\left(\Im M_{L+}^\mathrm{fit} - \Im M_{L+}^\mathrm{sol}\right)^2
\right]
\end{array}
\end{displaymath}
where
\begin{displaymath}
 |\mathcal{M}^\mathrm{sol}|^2 = \sum\limits_L^{L_\mathrm{max}}
\left(|E_{L-}^\mathrm{sol}|^2 + |E_{L+}^\mathrm{sol}|^2 + M_{L-}^\mathrm{sol}|^2 + |M_{L+}^\mathrm{sol}|^2\right)
\end{displaymath}
is used as a normalisation value which makes $Q_1$ dimensionless and independent from the absolute magnitude of the multipoles.
This ensures that the penalty contribution $Q_1$ will not vary significantly even when there are large changes in the size of
contributing multipoles (e.g. within or beyond the $\Delta(1232)$ region). $Q_1$ is also normalised
to the number of fit parameters $N_\mathrm{par}$, as the squared sum over real and imaginary parts up to $L_\mathrm{max}$
is proportional to the number of fitted multipoles (fixed multipoles always stay at their model values and hence do not
contribute to $Q_1$). Dividing out $N_\mathrm{par}$ and multiplying with the number of experimental data points
$N_\mathrm{pts}$ results in a $Q_1$ with the same scaling behaviour as $\chi^2$ (which is also proportional to $N_\mathrm{pts}$)
independent from the number of fit parameters.
The weight factor $q_1$ is globally adjustable and used to fine-tune the strength of the penalty contribution. Typical
values are around $q_1 = 1...2$.
As \textit{MLP1} considers the absolute deviations of multipoles it will mainly try to keep large multipoles close to their model values,
while small multipoles will be able to accumulate rather large relative deviations. So, when using \textit{MLP1} the
constraining is somewhat proportional to the size of the multipole. This may be not the best approach in cases where a single multipole
is vastly dominating, e.g. the $E_{0+}$ wave for $\eta$ production in the $S_{11}(1535)$ region.

\subsection{Penalty mode \textit{MLP2}}

The \textit{MLP2} penalty calculation evaluates a penalty term $Q_2$ from the squared sum of all relative deviations
between fitted 
(${E}_{L\pm}^\mathrm{fit}$, ${M}_{L\pm}^\mathrm{fit}$)
and model
(${E}_{L\pm}^\mathrm{sol}$, ${M}_{L\pm}^\mathrm{sol}$)
multipoles, according to
\begin{displaymath}
\begin{array}{lcl}
 Q_2 &=& q_2\cdot \dfrac{N_\mathrm{pts}}{N_\mathrm{par}} \cdot \left[
\sum\limits_L^{L_\mathrm{max}}
\frac{\left(\Re E_{L-}^\mathrm{fit} - \Re E_{L-}^\mathrm{sol}\right)^2}
{\left(\Re E_{L-}^\mathrm{sol}\right)^2 + \varepsilon^2}
+
\sum\limits_L^{L_\mathrm{max}}
\frac{\left(\Im E_{L-}^\mathrm{fit} - \Im E_{L-}^\mathrm{sol}\right)^2}
{\left(\Im E_{L-}^\mathrm{sol}\right)^2 + \varepsilon^2}\right.\\
&&\hspace{4.19em}+
\sum\limits_L^{L_\mathrm{max}}
\frac{\left(\Re E_{L+}^\mathrm{fit} - \Re E_{L+}^\mathrm{sol}\right)^2}
{\left(\Re E_{L+}^\mathrm{sol}\right)^2 + \varepsilon^2}
+
\sum\limits_L^{L_\mathrm{max}}
\frac{\left(\Im E_{L+}^\mathrm{fit} - \Im E_{L+}^\mathrm{sol}\right)^2}
{\left(\Im E_{L+}^\mathrm{sol}\right)^2 + \varepsilon^2}\\
&&\hspace{4.19em}+
\sum\limits_L^{L_\mathrm{max}}
\frac{\left(\Re M_{L-}^\mathrm{fit} - \Re M_{L-}^\mathrm{sol}\right)^2}
{\left(\Re M_{L-}^\mathrm{sol}\right)^2 + \varepsilon^2}
+
\sum\limits_L^{L_\mathrm{max}}
\frac{\left(\Im M_{L-}^\mathrm{fit} - \Im E_{L-}^\mathrm{sol}\right)^2}
{\left(\Im M_{L-}^\mathrm{sol}\right)^2 + \varepsilon^2}\\
&&\hspace{4.19em}+\left.
\sum\limits_L^{L_\mathrm{max}}
\frac{\left(\Re M_{L+}^\mathrm{fit} - \Re M_{L+}^\mathrm{sol}\right)^2}
{\left(\Re M_{L+}^\mathrm{sol}\right)^2 + \varepsilon^2}
+
\sum\limits_L^{L_\mathrm{max}}
\frac{\left(\Im M_{L+}^\mathrm{fit} - \Im E_{L+}^\mathrm{sol}\right)^2}
{\left(\Im M_{L+}^\mathrm{sol}\right)^2 + \varepsilon^2}
\right]
\end{array}
\end{displaymath}
with
$\varepsilon = \mbox{half the magnitude of the smallest non-zero real or imaginary part of all multipoles}$,
which basically represents the numerical precision of the used model values and is applied here to ensure
non-zero denominators even in cases of vanishing multipoles.
$Q_2$ is also normalised
to the number of fit parameters $N_\mathrm{par}$, as the squared sum over real and imaginary parts up to $L_\mathrm{max}$
is proportional to the number of fitted multipoles (fixed multipoles always stay at their model values and hence do not
contribute to $Q_2$). Dividing out $N_\mathrm{par}$ and multiplying with the number of experimental data points
$N_\mathrm{pts}$ results in a $Q_2$ with the same scaling behaviour as $\chi^2$ (which is also proportional to $N_\mathrm{pts}$)
independent from the number of fit parameters.
The weight factor $q_2$ is globally adjustable and used to fine-tune the strength of the penalty contribution.
Values for $q_2$ should be significantly smaller than for $q_1$ in the \textit{MLP1} mode, a typical number
would be $q_2 \simeq 0.5$.
As \textit{MLP2} considers relative deviations, large multipoles will be much less constrained compared to \textit{MLP1}.
This might be difficult if small multipoles that are not very well described by the used model
will contribute significantly to the penalty and affect larger, well-known multipoles.

\subsection{Penalty mode \textit{MLP3}}

The \textit{MLP3} penalty calculation is a combination of \textit{MLP1} and \textit{MLP2} modes and combines both
penalties based on absolute and relative deviations. Consequently, the \textit{MLP3} penalty is calculated according to
\begin{displaymath}
Q_3 = Q_1 + Q_2
\end{displaymath}
The individual weight factors $q_1$ and $q_2$ will be used here to adjust the balance between absolute and relative
constraints to the fitted multipoles. A careful selection of $q_1$ and $q_2$ could therefore mitigate 
the dominance effects of very large and very small multipoles for \textit{MLP1} and \textit{MLP2} modes.

\subsection{Penalty mode \textit{CGLN}}

The \textit{CGLN} mode calculates a penalty term from the difference between the four
CGLN amplitudes $F_i^\mathrm{fit}$ obtained from current fit parameters and the model values $F_i^\mathrm{sol}$,
according to
\begin{displaymath}
 Q_4 =\frac{1}{|\mathcal{F}^\mathrm{sol}|^2}
\sum_{i=1}^4
\sum_{j=1}^{N_\mathrm{pts}}
\frac{|F_i(x_j)^\mathrm{fit} - F_i(x_j)^\mathrm{sol}|^2}{f_i^2}
\end{displaymath}
where
\begin{displaymath}
 |\mathcal{F}^\mathrm{sol}|^2 = \sum_{i=1}^4\sum_{j=1}^{N_\mathrm{pts}} |F_i(x_j)^\mathrm{sol}|^2
\end{displaymath}
is used as a normalisation value which makes $Q_4$ dimensionless and independent from the absolute magnitude of the amplitudes.
This ensures that the penalty contribution $Q_4$ will not vary significantly even when there are large changes in the size of
the amplitudes (e.g. within or beyond the $\Delta(1232)$ region).
CGLN amplitudes are evaluated at $N_\mathrm{pts}$ (the number of experimental data points from all available observables)
positions for $x_j = \cos\theta_j$ which are distributed equidistant in the range $x = [-1,1]$.
Evaluating the CGLN amplitudes at $N_\mathrm{pts}$ polar angle positions makes sure that $Q_4$ has the same scaling behaviour
as $\chi^2$ (which is also proportional to $N_\mathrm{pts}$).
For each CGLN amplitude $F_i$ an individual weight $f_i$ is applied in the squared sum of deviations. These weights are used
to adjust the size of the penalty contribution $Q_4$ relative to $\chi^2$. Note that the penalty term $Q_4$ depends on
four external parameters which allows a high level of flexibility.

\subsection{Penalty mode \textit{HELI}}

The \textit{HELI} mode is basically very similar to the \textit{CGLN} mode, with the only difference being the
usage of helicity amplitudes $H_i$ instead of CGLN amplitudes $F_i$. For the calculation of helicity amplitudes
the transformation relations
\begin{displaymath}\label{frm_Heli}
\begin{array}{l}
H_1 =           -\dfrac{1}{\sqrt{2}}\sin\theta\cos\frac{\theta}{2} \left(F_3 + F_4\right) \\
H_2 = \phantom{-}\sqrt{2}\cos\frac{\theta}{2}\left[\left(F_2 - F_1\right) + \frac{1-\cos\theta}{2}\left(F_3 - F_4\right)\right] \\
H_3 = \phantom{-}\dfrac{1}{\sqrt{2}}\sin\theta\sin\frac{\theta}{2} \left(F_3 - F_4\right) \\
H_4 = \phantom{-}\sqrt{2}\sin\frac{\theta}{2}\left[\left(F_1 + F_1\right) + \frac{1+\cos\theta}{2}\left(F_3 + F_4\right)\right]
\end{array}
\end{displaymath}
between $H_i$ and $F_i$ are used and the penalty term is given by
\begin{displaymath}
 Q_5 =\frac{1}{|\mathcal{H}^\mathrm{sol}|^2}
\sum_{i=1}^4
\sum_{j=1}^{N_\mathrm{pts}}
\frac{|H_i(x_j)^\mathrm{fit} - H_i(x_j)^\mathrm{sol}|^2}{h_i^2}
\end{displaymath}
with a normalisation value
\begin{displaymath}
 |\mathcal{H}^\mathrm{sol}|^2 = \sum_{i=1}^4\sum_{j=1}^{N_\mathrm{pts}} |H_i(x_j)^\mathrm{sol}|^2
\end{displaymath}
making $Q_5$ dimensionless and independent from the absolute magnitude of the amplitudes. Also in this mode amplitudes
are evaluated at equidistant $x_j$ positions covering the full angular range. Similar to \textit{CGLN} mode
the penalty contribution is adjusted by individual weights $h_i$ for each helicity amplitude $H_i$.

\section{SE fits and observable energies}

\PWA performs multipole fits on fixed energy positions which are (with rare exceptions) given by the energy information of
unpolarised cross section $\sigma_0$ data. Such $\sigma_0$ measurements have typically the finest energy binning as well as
the highest statistical precision, while polarisation observable data mostly is only available in much wider energy bins and
with lower statistics.
Up to now \PWA does not perform any adjustments of energy positions or interpolation schemes for any observables
but determines multipoles at the \textbf{exact} energies of the $\sigma_0$ dataset, while data points for 
polarisation observables are considered in the fit, if the current $\sigma_0$ energy position is covered
by the respective polarisation observable energy bin. This means that a single value for asymmetry observables like
$\Sigma$, $T$, $P$, $E$, $F$, $G$, $H$ which are given in energy bins of typically $20...100$\:MeV will be used
for fits at multiple $\sigma_0$ energy positions.
The rationale for this scheme is that any energy mismatch between the SE fit energy and the $\sigma_0$ energy
would have a large impact on the fit quality due to the high precision of cross section data, while 
the larger uncertainties for polarisation observables provide some insensitivity to such energy deviations.
Also keep in mind that a polarisation observable determined for a $\sim\hspace{-0.4em}50$\:MeV bin always represents an average
value of all energies within that bin, which justifies the repeated use of such data points for multiple
cross section energies that are compatible with the polarisation observable bin.
In the long term a more advanced handling of observable energy mismatches might be desirable,
however such interpolation schemes are still a rather open question and several different approaches 
are currently discussed.

\section{Fits with variation of start parameters}

Multipole fits with high orders $L_\mathrm{max}$ operate in a high-dimensional parameter space and
any found minimum in the $f_\mathrm{FCN}$ funktion is not necessarily the global minimum
representing the physically correct solution. In order to search for possible multiple
solutions for an SE fit \PWA can be configured in a way that it repeats the fit process with
slight variations in the start parameters, so that it probably finds a different $\chi^2$ minimum.
Such start parameters for the fitting process are typically taken from any existing photoproduction model or
partial wave analysis. \PWA changes these start parameters for all fittable parameters by random numbers
from within an adjustable band around the model solution (fixed multipoles that are parametrised in the fit process
always stay at their exact model values) and repeats every SE fit several times. Unique solutions 
are identified using their $f_\mathrm{FCN}$ value and a configurable number of found solutions is stored. 
At the moment \PWA does \textbf{no} further processing on these different solutions, like selecting solutions which 
have a smooth energy dependence with respect to their neighbouring energy points.

\section{Observable rescaling}

During fits \PWA has the possibility to rescale observable datasets (that is all data points for a given energy)
within given systematic uncertainties to achieve a better agreement between
different datasets. Technically this is done by additional fit parameters $s_\Omega$ for scaling each
observable and a penalty term, which penalises any deviation from the nominal values. This penalty term and the modified $\chi^2$
contribution (including the scaling parameter) for an observable $\Omega$ (cross sections and asymmetries) are given by
\begin{displaymath}
 Q_s = q_s\cdot \sum\limits_\Omega\left[ N_{\mathrm{pts},\Omega} \cdot \left(\frac{s_\Omega - 1}{\delta_\Omega}\right)^2\right]
\end{displaymath}
and
\begin{displaymath}
\chi_\Omega^2
= \sum\limits_i \left(
\frac
{s_\Omega \Omega^\mathrm{exp}(\theta_i) - \Omega^\mathrm{fit}(\theta_i)}
{s_\Omega\Delta \Omega^\mathrm{exp}(\theta_i)}
\right)^2
\end{displaymath}
The relative systematic uncertainty (e.g. $5\% = 0.05$) is given by $\delta_\Omega$, while $N_{\mathrm{pts},\Omega}$
is the number of data points for observable $\Omega$. Making the individual contributions proportional
to  $N_{\mathrm{pts},\Omega}$ ensures that rescaling observables with many data points gets more `expensive'
than observables with only very few points, and provides a similar scaling behaviour
for $Q_s$ as  for $\chi^2$ (which is proportional to $N_\mathrm{pts} = \sum N_{\mathrm{pts},\Omega} $).
The additional global weight factor $q_s$ can be used to adjust the strength of the rescaling. Small values for
$q_s$ decrease the penalty contribution and hence give the fitting process more freedom to rescale observables.

\section{\PWA installation}

The current version of the Mainz \PWA package can be obtained from its GitHub repository available under\\
\tt https://github.com/MainzPWA/PWA\rm \\
\PWA will run under Linux, FreeBSD, and OS X systems and can be compiled on respective systems using the provided
scripts \tt build.linux\rm, \tt build.freebsd\rm, and \tt build.macosx\rm. \PWA requires a (recent) \tt ROOT\rm\ installation.
For successful compilation it might be necessary to adjust the variables \tt \$ROOT\_LIB\rm\ and \tt \$ROOT\_INC\rm\
in the build scripts to match the directories with libraries and header files of your \tt ROOT\rm\ installation.
In addition, a \tt CMakeLists.txt\rm\ file for use with the \tt cmake\rm\ build system is provided.

\section{Setting up a work directory for \PWA}

It is recommended to set up a special work directory for your \PWA SE fits. This directory should contain
all necessary experimental data, model inputs used for the fits, and a configuration file for \tt PWA\rm.
A typical work directory consists of the follwing entries:
\begin{itemize}
\item A directory \tt model\rm\ containing model values for electromagnetic multipoles, which are used
as start parameters and for penalty calculations during the fitting process.
It is recommended to have a \tt model\rm\ directory as a symbolic link to one of the models that \PWA provides within 
its package, e.g. use the command\\
\tt ln -s /home/\textit{username}/PWA/model/ppi0/MAID model\rm\\
to use MAID calculations for your $p \pi^0$ SE fits. 
As the model path has to be given in the \PWA configuration file, it is not required to have a dedicated \tt model\rm\ directory, but
not having absolute paths in the \PWA configuration file can simplify things.

\item A directory \tt data\rm\ containing experimental datasets for observables you want to use in your fit.
It is recommended to have a \tt data\rm\ directory as a symbolic link to the datasets that \PWA provides within 
its package, e.g. use the command\\
\tt ln -s /home/\textit{username}/PWA/data/ppi0 data\rm\\
This will make various $p\pi^0$ datasets accessible to your work directory. As filenames for all datasets you want to use for fits
have to be given in the \PWA configuration file, it is not required to have a dedicated \tt data\rm\ directory, but
not having absolute paths in the \PWA configuration file can simplify things.

\item A \tt PWA.cfg\rm\ file, holding various configuration options (see section \ref{sc_config}).
\PWA by default tries to read its configuration from \tt PWA.cfg\rm\ in the current directory, but a different filename
can be passed as command-line parameter, e.g.\\
\PWA \tt /home/\textit{username}/cfg/fit.cfg\rm

\item The \PWA executable. Here it is recommended to either create a (symbolic) link to the actual \PWA binary
which normally can be found under \tt build/bin/PWA\rm\ or to include \PWA  in your search path.
\end{itemize}
During the fits \PWA will create additional \tt plots.\textit{n}\rm\ directories holding the fit results.
The number of \tt plots.\textit{n}\rm\ directories depends on the amount of solutions you want \PWA to store.
An example of such a work directory for $p \pi^0$ data and fits based on MAID can be found in
\tt work/ppi0/MAID\rm\ in the main directory of the \PWA package.

\section{Configuration options}\label{sc_config}

The following sections describe the various configuration options for fits with \PWA.
Options are generally read from the configuration file \tt PWA.cfg\rm\ in your \PWA work directory, if \PWA
is invoked without additional command-line parameters. However, a different configuration file may be passed as parameter,
e.g.\\
\PWA \tt fit.cfg\rm\\
 All following configuration keywords are case-sensitive, which makes
\tt OPTION\rm\ different from \tt option\rm, where all keywords have to  be upper case.
Any non-existing keyword will be ignored, therefore no explicit markup for comments is required.

\subsection{Kinematic parameters}

\begin{itemize}
\item
\tt MASS\_MESON \textit{mass}\rm\\
Sets the mass of the produced final state meson ($\pi^0$, $\pi^\pm$, $\eta$, $\eta^\prime$, $K^\pm$, ...) to the given value
 (in MeV). This parameter is mandatory for correct calculation of kinematic factors during the fitting process.
Examples are
\vspace{-0.5em}\begin{itemize}
 \item[$\pi^0$\:]          $134.9766$
 \item[$\pi^+$]            $139.5702$
 \item[$\eta\phantom{^+}$] $547.8530$
\end{itemize}

\tt MASS\_INITIAL \textit{mass}\rm\\
Sets the mass of the inital state baryon ($p$, $n$) to the given value
 (in MeV). This parameter is mandatory for correct calculation of kinematic factors during the fitting process.
Examples are
\vspace{-0.5em}\begin{itemize}
 \item[$p$\:\:] $938.2720$
 \item[$n$\:\:] $939.5654$
\end{itemize}

\tt MASS\_FINAL \textit{mass}\rm\\
Sets the mass of the produced final state baryon ($p$, $n$, $\Lambda$, $\Sigma$, ...) to the given value
 (in MeV). This parameter is mandatory for correct calculation of kinematic factors during the fitting process.
Examples are
\vspace{-0.5em}\begin{itemize}
 \item[$p$\:\:] $\phantom{1}938.2720$
 \item[$n$\:\:] $\phantom{1}939.5654$
 \item[$\Lambda$\:\:]     $1115.6830$
\end{itemize}
\end{itemize}

\subsection{Fitted multipoles}

\begin{itemize}
\item
\tt L\_MAX \textit{n}\rm\\
This sets the maximum multipole order $L_\mathrm{max}$ (which is considered in the calculation of CGLN amplitudes and observables)
to the given value \tt\textit{n}\rm. Currently \PWA supports multipole orders up to $L_\mathrm{max} = 10$. 
Note that this does \textbf{not} set the maximum multipole order which is actually fitted, as the fittable multipoles
can be selected indiviually later on. Any multipole with $L \leq L_\mathrm{max}$ that is not fitted will enter
the calculation of CGLN amplitudes and observables as a parametrisation using the current model values.

\item
\tt FIX\_E\textit{l}P 0/1\rm\\
\tt FIX\_M\textit{l}P 0/1\rm\\
\tt FIX\_E\textit{l}M 0/1\rm\\
\tt FIX\_M\textit{l}M 0/1\rm\\
Any $E_{l+}$, $M_{l+}$, $E_{l-}$, $M_{l-}$ multipole can either be fixed (\tt1\rm) to its model values
or be used as fit parameters (\tt0\rm).
Fixing a multipole here fixes both magnitude and phase (or real and imaginary values).
Hence any fittable multipole will add two parameters to the fitting process.

\item
\tt FIX\_E\textit{l}P\_PHASE 0/1\rm\\
\tt FIX\_E\textit{l}M\_PHASE 0/1\rm\\
\tt FIX\_M\textit{l}P\_PHASE 0/1\rm\\
\tt FIX\_M\textit{l}M\_PHASE 0/1\rm\\
For any $E_{l+}$, $M_{l+}$, $E_{l-}$, $M_{l-}$ multipole that was previously set as fittable the individual
phase can be either fixed (\tt1\rm) to its model value or be used as fit parameter (\tt0\rm). Fixing a phase
will reduce the number of fit parameters by one and only fit the magnitude for the given multipole.

\end{itemize}

\subsection{Penalty options}

\begin{itemize}
\item
\tt PENALTY\_MODE \textit{n}\rm\\
Selects a penalty mode for use during the minimisation process. Possible values for $\enn$ are
\vspace{-0.5em}\begin{enumerate}
\item[0\:\:] No penalty contribution
\item[1\:\:] \textit{MLP1}\hspace{0.16em} penalty mode
\item[2\:\:] \textit{MLP2}\hspace{0.16em} penalty mode
\item[3\:\:] \textit{MLP3}\hspace{0.17em} penalty mode
\item[4\:\:] \textit{CGLN} penalty mode
\item[5\:\:] \textit{HELI}\hspace{0.43em} penalty mode
\end{enumerate}

\item
\tt PENALTY\_MLP1 \textit{value}\rm\\
Sets the weight factor $q_1$ to the given value. This value will be used for for \textit{MLP1} and \textit{MLP3} penalty modes.
Typical values are around $q_1 = 2.0$.
\item
\tt PENALTY\_MLP2 \textit{value}\rm\\
Sets the weight factor $q_2$ to the given value. This value will be used for for \textit{MLP2} and \textit{MLP3} penalty modes.
Typical values are around $q_2 = 0.5$.

\item
\tt WEIGHT\_A\textit{n} \textit{value}\rm\\
Sets the individual weight factors for CGLN or helicity amplitude $\mbox{\tt\textit{n}\rm} = 1...4$ to the given value. These values will be used
in CGLN or HELI penalty mode only. Typical values are around $f_n = 1.0$.

\item
\tt PRINT\_PENALTY 0/1\rm\\
When enabled (\tt1\rm) an additional line will be printed out for each fit giving the individual contributions of $\chi^2$ and
penalty $Q$ for the found solution.

\end{itemize}

\subsection{Error calculation}

\begin{itemize}
\item
\tt ERROR\_MODE \textit{n}\rm\\
Selects a minimisation function for error calculation through MINUIT. Possible options for $\enn$ are
\vspace{-0.5em}\begin{enumerate}
\item[1\:\:] Use only $\chi^2$ function for error determination. This is the mathematically most sound approach, as the definition of
parameter error ranges is based on the variation of $\chi^2 \rightarrow \chi^2 +1$. However, for a
minimisation with penalty contributions, the minima for $\chi^2$ only and $\chi^2+Q$ are not necessarily
identical and the behaviour of $\chi^2$ only around the chosen $\chi^2+Q$ minimum might result in ill-defined parameter
errors\footnote{When varying $\chi^2$ around a non-minimum position the first derivations of parameters might be negative
(indicating the possibility to find a lower $\chi^2$ nearby). So the error matrix is not necessarily always positive definite.
MINUIT tries to estimate errors in such cases, but the results are probably not fully reliable.}.
\item[2\:\:] Use sum of $\chi^2$ and current penalty $Q$ for error determination. This might in general overestimate
parameter errors but will always produce well-defined results.
\item[3\:\:] Use an adaptive mode that prefers the $\chi^2$ only calculation but falls back to the 
sum of $\chi^2$ and current penalty $Q$ if this produces smaller errors. This is the recommended mode.
\end{enumerate}

\end{itemize}

\subsection{Multiple fit solutions}

\begin{itemize}
\item
\tt ITERATIONS \textit{n}\rm\\
Individual SE fits will be repeated $\enn$ times with small variations in the starting parameters to find possible additional
minima in the parameter space. When the given number of iterations has been processed the best solution(s), i.e.
with lowest $\chi^2+Q$, are selected. Computing time for the fitting process increases linear with the number of iterations.
For reasonable performance, $\enn$ should not exceed values of $50...100$.

\item
\tt SOLUTIONS \textit{n}\rm\\
Up to $\enn$ unique solutions that have been found during the iterations for each SE fit will be stored (with decreasing fit quality).
The number of maximum solutions to store may not exceed the number of iterations for the fit process.

\item
\tt VARIATION\_REL \textit{value}\rm\\
This will define a \textbf{relative} variation of start parameters for each iteration equally distributed
in a band around its current model value. The relative size of this band is given by $\pm$\tt\textit{value}\rm. A typical value
is $0.2$ which will result in parameter variations by $\pm20\%$.  Note that relative and absolute variations will be added.

\item
\tt VARIATION\_ABS \textit{value}\rm\\
This will define an \textbf{absolute} variation of start parameters for each iteration equally distributed
in a band around its current model value. The size of this band is given by $\pm$\tt\textit{value}\rm\ in units of $10^{-3}/m_{\pi^+}$.
This option is useful to define an absolute minimum of the variation band, which will be effective for very small
multipoles (where a relative variation would have only a small effect). Note that relative and absolute variations will be added.
\end{itemize}

\subsection{Energy range for fits}

\begin{itemize}
\item
\tt MIN\_ENERGY \textit{value}\rm\\
\tt MAX\_ENERGY \textit{value}\rm\\
These options define the range of photon beam energies $\omega$ (in MeV) for which fits are performed.
\end{itemize}

\subsection{Observable rescaling}

\begin{itemize}
\item
\tt FIX\_SCALES 0/1\rm\\
Experimental observables can be fixed to their actual values (\tt1\rm) or scaled within their given systematic uncertainty ranges (\tt0\rm)
during the fit process. This will introduce one additional fit parameter (scaling value) for each observable.
Note that $X$ and $\sigma_X$ are considered here as different observables.

\item
\tt SCALING \textit{value}\rm\\
This is an additional weight factor that is applied to the penalty contribution imposed by the scaling factor variation.
A small \tt\textit{value}\rm\ (e.g. 0.1) will decrease this contribution and therefore
give the fitter more freedom to rescale experimental values.
\end{itemize}

\subsection{Configuration of experimental data}

\begin{itemize}
\item
\tt SG0\_FILE\bl\bl \textit{Path/Filename}\bl \textit{weight}\bl \textit{scale}\rm\\
\tt SGS\_FILE\bl\bl \textit{Path/Filename}\bl \textit{weight}\bl \textit{scale}\rm\\
\tt SGT\_FILE\bl\bl \textit{Path/Filename}\bl \textit{weight}\bl \textit{scale}\rm\\
\tt SGP\_FILE\bl\bl \textit{Path/Filename}\bl \textit{weight}\bl \textit{scale}\rm\\
\tt SGE\_FILE\bl\bl \textit{Path/Filename}\bl \textit{weight}\bl \textit{scale}\rm\\
\tt SGF\_FILE\bl\bl \textit{Path/Filename}\bl \textit{weight}\bl \textit{scale}\rm\\
\tt SGG\_FILE\bl\bl \textit{Path/Filename}\bl \textit{weight}\bl \textit{scale}\rm\\
\tt SGH\_FILE\bl\bl \textit{Path/Filename}\bl \textit{weight}\bl \textit{scale}\rm\\
\tt SGCX\_FILE\bl   \textit{Path/Filename}\bl \textit{weight}\bl \textit{scale}\rm\\
\tt SGCZ\_FILE\bl   \textit{Path/Filename}\bl \textit{weight}\bl \textit{scale}\rm\\
\tt SGOX\_FILE\bl   \textit{Path/Filename}\bl \textit{weight}\bl \textit{scale}\rm\\
\tt SGOZ\_FILE\bl   \textit{Path/Filename}\bl \textit{weight}\bl \textit{scale}\rm\\
\tt SGLX\_FILE\bl   \textit{Path/Filename}\bl \textit{weight}\bl \textit{scale}\rm\\
\tt SGLZ\_FILE\bl   \textit{Path/Filename}\bl \textit{weight}\bl \textit{scale}\rm\\
\tt SGTX\_FILE\bl   \textit{Path/Filename}\bl \textit{weight}\bl \textit{scale}\rm\\
\tt SGTZ\_FILE\bl   \textit{Path/Filename}\bl \textit{weight}\bl \textit{scale}\rm\\
Defines the experimental datasets for cross sections (unpolarised $\sigma_0$ and polarised $\sigma_X$). The additional
parameter \tt\textit{weight}\rm\ is applied to any $\chi^2$ contributions from the corresponding dataset. The additional
parameter \tt\textit{scale}\rm\ defines an overall scaling factor to the corresponding data and errors, i.e.
a value of $\mbox{\tt\textit{scale}\rm} = 1.05$ will increase the data by 5\%.
Multiple lines for observables with more than one dataset are possible (with individual values for
\tt\textit{weight}\rm\ and \tt\textit{scale}\rm).
The parameters \tt\textit{weight}\rm\ and \tt\textit{scale}\rm\ must not be omitted.

\item
\tt S\_FILE\bl\bl \textit{Path/Filename}\bl \textit{weight}\bl \textit{scale}\rm\\
\tt T\_FILE\bl\bl \textit{Path/Filename}\bl \textit{weight}\bl \textit{scale}\rm\\
\tt P\_FILE\bl\bl \textit{Path/Filename}\bl \textit{weight}\bl \textit{scale}\rm\\
\tt E\_FILE\bl\bl \textit{Path/Filename}\bl \textit{weight}\bl \textit{scale}\rm\\
\tt F\_FILE\bl\bl \textit{Path/Filename}\bl \textit{weight}\bl \textit{scale}\rm\\
\tt G\_FILE\bl\bl \textit{Path/Filename}\bl \textit{weight}\bl \textit{scale}\rm\\
\tt H\_FILE\bl\bl \textit{Path/Filename}\bl \textit{weight}\bl \textit{scale}\rm\\
\tt CX\_FILE\bl   \textit{Path/Filename}\bl \textit{weight}\bl \textit{scale}\rm\\
\tt CZ\_FILE\bl   \textit{Path/Filename}\bl \textit{weight}\bl \textit{scale}\rm\\
\tt OX\_FILE\bl   \textit{Path/Filename}\bl \textit{weight}\bl \textit{scale}\rm\\
\tt OZ\_FILE\bl   \textit{Path/Filename}\bl \textit{weight}\bl \textit{scale}\rm\\
\tt LX\_FILE\bl   \textit{Path/Filename}\bl \textit{weight}\bl \textit{scale}\rm\\
\tt LZ\_FILE\bl   \textit{Path/Filename}\bl \textit{weight}\bl \textit{scale}\rm\\
\tt TX\_FILE\bl   \textit{Path/Filename}\bl \textit{weight}\bl \textit{scale}\rm\\
\tt TZ\_FILE\bl   \textit{Path/Filename}\bl \textit{weight}\bl \textit{scale}\rm\\
Defines the experimental datasets for asymmetry observables. The additional
parameter \tt\textit{weight}\rm\ is applied to any $\chi^2$ contributions from the corresponding dataset. The additional
parameter \tt\textit{scale}\rm\ defines an overall scaling factor to the corresponding data and errors, i.e.
a value of $\mbox{\tt\textit{scale}\rm} = 0.95$ will decrease the data by 5\%.
Multiple lines for observables with more than one dataset are possible (with individual values for
\tt\textit{weight}\rm\ and \tt\textit{scale}\rm).
The parameters \tt\textit{weight}\rm\ and \tt\textit{scale}\rm\ must not be omitted.

\item
\tt USE\_PRELIMINARY 0/1\rm\\
Defines whether datasets tagged as `preliminary' in their respective files should be used (\tt1\rm) or not (\tt0\rm).
With this option preliminary datasets need not to be removed from the 
\tt SG\textit{X}\_FILE\rm\ and \tt \textit{X}\_FILE\rm\ lists but their data will be ignored.
\end{itemize}

\subsection{Configuration of model values}

\begin{itemize}
\item
\tt MODEL\_PATH \textit{Path}\rm\\
Defines the directory containing theoretical values for multipoles. The given directory must hold text files
following the naming scheme \tt E\textit{l}p.txt\rm,
\tt E\textit{l}m.txt\rm\ 
for electric multipoles, and
\tt M\textit{l}p.txt\rm,
\tt M\textit{l}m.txt\rm\ 
for magnetic multipoles, with angular momentum \tt\textit{l}\rm, and parity 
$+$ (\tt p\rm) or $-$ (\tt m\rm).

\end{itemize}

\subsection{Special options for $\pi^0$ threshold fits}

\begin{itemize}
\item
\tt ONLY\_CROSS\_S 0/1\rm\\
\tt ONLY\_CROSS\_F 0/1\rm\\
These options restrict the data used for fitting to $\lbrace\sigma_0,\sigma_\Sigma, \Sigma\rbrace$ or
$\lbrace\sigma_0,\sigma_F, F\rbrace$. Any other observables are ignored.
These options are mutually exclusive. Also \tt FIX\_E\textit{n}P\_PHASE\rm\ must be set to \tt0\rm\
for all $s$ and $p$ wave multipoles.

\tt FIX\_RE\_E0P 0/1\rm\\
The real part of the $E_{0+}$ multipole can either be fixed (\tt1\rm) to its model value or be used as a fit parameter (\tt0\rm).
In order to use this option, both \tt FIX\_E0P\rm\ and \tt FIX\_E0P\_PHASE\rm\ must be set to \tt0\rm.

\tt FIX\_IM\_E0P 0/1\rm\\
The imaginary part of the $E_{0+}$ multipole can either be fixed (\tt1\rm) or be used as a fit parameter (\tt0\rm). If $\Im E_{0+}$
is fixed, a parametrisation according to $\Im E_{0+} = \beta \cdot \frac{q_{\pi^+}}{m_{\pi^+}}$ is used.
Note that \tt FIX\_IM\_E0P\rm\ can only be used in combination with \tt ONLY\_CROSS\_S\rm\
or \tt ONLY\_CROSS\_F\rm. Also 
both \tt FIX\_E0P\rm\ and \tt FIX\_E0P\_PHASE\rm\ must be set to \tt0\rm.

\item
\tt BETA \textit{value}\rm\\
This option sets the value for $\beta$ used in the parametrisation of $\Im E_{0+}$. Typical values are
$\beta = 3.43$ (with isospin symmetry) or $\beta = 3.35$ (with isospin breaking).

\item
\tt D\_WAVES \textit{n}\rm\\
Selects which contributions for $d$ waves is used (if supported by the current model). Possible options are
\vspace{-0.5em}\begin{enumerate}
\item[1\:\:] Full model calculation
\item[2\:\:] Born terms only
\item[3\:\:] Born terms and $\rho/\omega$ exchange
\end{enumerate}
\vspace{-0.5em}Right now, only MAID and DMT provide these three different $d$ wave contributions.

\item
\tt SGT\_ENERGIES 0/1\rm\\
With this option enabled (\tt1\rm) SE fits are performed at energies given by $\sigma_T$ data instead of $\sigma_0$.
\end{itemize}

\section{Fitting with \PWA}

To perform SE fits with \PWA simply execute\\
\PWA \tt [\textit{PWA.cfg}] \rm \\
in your work directory. For each SE fit position \PWA will output a summary like
\begin{verbatim}
------------------------------------------------------------------------------------
omega = 727.200 MeV, W = 1498.325 MeV
Mlp  |  Fit (Chi^2/NDF =  0.828, NDF =  23)       |  Model               |  Chi'^2
-----+--------------------------------------------+----------------------+----------
E0+  |  ( -5.440 +- 0.469) + ( -7.175 +- 0.618)i  |   -8.866 + -11.695i  |   53.4906
E1+  |  (  1.963 +- 0.319) + (  0.542 +- 0.360)i  |   -0.004 +  -0.044i  |   18.2068
M1+  |  (  1.908 +- 0.317) + (  1.044 +- 0.419)i  |    0.014 +  -0.034i  |   17.2096
M1-  |  ( -1.410 +- 0.799) + (  7.377 +- 0.826)i  |    0.231 +  -0.315i  |   46.8098
E2+  |  ( -0.096 +- 0.192) + (  0.313 +- 0.147)i  |   -0.001 +   0.000i  |    1.8285
E2-  |  (  0.271 +- 0.326) + ( -0.132 +- 0.397)i  |   -0.095 +  -0.160i  |    0.5122
M2+  |  ( -0.308 +- 0.179) + (  0.270 +- 0.165)i  |    0.008 +  -0.006i  |    2.9562
M2-  |  ( -0.174 +- 0.293) + ( -0.716 +- 0.266)i  |   -0.040 +  -0.088i  |    2.6338
E3+  |  ( -0.571 +- 0.105) + (  0.378 +- 0.080)i  |   -0.000 +  -0.000i  |   26.7770
E3-  |  ( -0.100 +- 0.139) + ( -0.252 +- 0.142)i  |   -0.002 +   0.000i  |    1.8658
M3+  |  ( -0.437 +- 0.098) + (  0.353 +- 0.098)i  |   -0.000 +  -0.001i  |   16.3666
M3-  |  (  0.056 +- 0.137) + (  0.373 +- 0.129)i  |   -0.001 +  -0.000i  |    3.9924
E4+  |  ( -0.312 +- 0.057) + ( -0.097 +- 0.043)i  |    0.000 +   0.000i  |   20.8915
E4-  |  (  0.094 +- 0.046) + ( -0.134 +- 0.073)i  |   -0.000 +   0.000i  |    3.6069
M4+  |  ( -0.324 +- 0.053) + ( -0.050 +- 0.046)i  |   -0.000 +  -0.000i  |   21.6926
M4-  |  ( -0.030 +- 0.041) + (  0.068 +- 0.078)i  |   -0.000 +  -0.000i  |    0.7070
E5+  |  (  0.000 +- 0.000) + (  0.000 +- 0.000)i  |    0.000 +  -0.000i  |    0.0000
E5-  |  (  0.000 +- 0.000) + (  0.000 +- 0.000)i  |    0.000 +  -0.000i  |    0.0000
M5+  |  (  0.000 +- 0.000) + ( -0.000 +- 0.000)i  |    0.000 +  -0.000i  |    0.0000
M5-  |  ( -0.000 +- 0.000) + (  0.000 +- 0.000)i  |   -0.000 +   0.000i  |    0.0000
-----+--------------------------------------------+----------------------+----------
Chi^2 =   19.03, Penalty =    0.00, Scale =  0.00, NPts =  54, NPar = 31, NDF =  23
------------------------------------------------------------------------------------
\end{verbatim}
The last column quantifies the deviation of the fit result compared to the used model prediction 
for each multipole $\mathcal{M}_{L\pm}$ according to
\begin{displaymath}
 \chi^{\prime2} = \frac{|\mathcal{M}_{L\pm}^\mathrm{sol} - \mathcal{M}_{L\pm}^\mathrm{fit}|^2}{|\Delta\mathcal{M}_{L\pm}^\mathrm{fit}|^2}
\end{displaymath}
The last line, which only occurs if the configuration option \tt PRINT\_PENALTY\rm\ is set, summarises
the different contributions ($\chi^2$, penalty $Q_{1...5}$, observable scaling penalty $Q_s$)
to the $f_\mathrm{FCN}$ funtion, as well as the number of experimental data points ($N_\mathrm{pts}$) for the current energy,
the number of fit parameters ($N_\mathrm{par}$), and the resulting degrees of freedom (\textit{NDF}).
The above example was executed with observable rescaling disabled and without penalty.
If you also want to store these summary printouts, it is recommended to use the output redirection of your shell, e.g. 
execute\\
\PWA \tt [\textit{PWA.cfg}] > PWA.out \rm \\
or\\
\PWA \tt [\textit{PWA.cfg}]  | tee PWA.out \rm \\
The second version has the advantage that the \PWA output is written to both the screen and the output file \tt PWA.out\rm.
In addition to these summary outputs \PWA will also create \tt.root\rm\ files with results for each multipole that can be used
for further processing (see sections \ref{sec_output} and \ref{sec_macros}).

\section{Data formats}

The following sections describe the different text data formats used by \PWA for experimental observable data and model multipole values. 

\subsection{Experimental observable data}

Experimental data is organised in separate files for each observable (asymmetries $X$ and polarised cross sections $\sigma_X$
are considered here as different observables) \textbf{and} for each individual measurement. A file for such an individual experiment
however can contain data for more than one energy.
Such an entry for an energy bin consists of general information on the current data as well as of the actual observable data
for different polar angle positions. A typical file looks like the following:\\

\tt
E =  683.50 MeV, E\_lo =  667.24 MeV, E\_hi =  699.76 MeV\\
Systematic = 0.0248, Preliminary = 0, CBELSA\_2014\_Hartmann\_PRL113-062001\\
163.335\bl\bl\bl -0.2596\bl\bl\bl 0.0976\bl\bl\bl 0.0080\\
151.045\bl\bl\bl -0.3346\bl\bl\bl 0.0543\bl\bl\bl 0.0086\\
142.373\bl\bl\bl -0.4980\bl\bl\bl 0.0408\bl\bl\bl 0.0129\\
135.072\bl\bl\bl -0.6306\bl\bl\bl 0.0347\bl\bl\bl 0.0154\\
128.682\bl\bl\bl -0.6693\bl\bl\bl 0.0316\bl\bl\bl 0.0168\\
122.820\bl\bl\bl -0.7317\bl\bl\bl 0.0286\bl\bl\bl 0.0170\\
117.258\bl\bl\bl -0.8295\bl\bl\bl 0.0338\bl\bl\bl 0.0261\\
112.024\bl\bl\bl -0.7583\bl\bl\bl 0.0291\bl\bl\bl 0.0170\\
106.978\bl\bl\bl -0.7793\bl\bl\bl 0.0267\bl\bl\bl 0.0180\\
102.005\bl\bl\bl -0.7629\bl\bl\bl 0.0263\bl\bl\bl 0.0189\\
\phantom{1}97.181\bl\bl\bl -0.7808\bl\bl\bl 0.0249\bl\bl\bl 0.0179\\
\phantom{1}92.407\bl\bl\bl -0.7951\bl\bl\bl 0.0247\bl\bl\bl 0.0193\\
\phantom{1}87.593\bl\bl\bl -0.7840\bl\bl\bl 0.0251\bl\bl\bl 0.0190\\
\phantom{1}82.819\bl\bl\bl -0.7738\bl\bl\bl 0.0260\bl\bl\bl 0.0179\\
\phantom{1}77.995\bl\bl\bl -0.7537\bl\bl\bl 0.0269\bl\bl\bl 0.0188\\
\phantom{1}73.022\bl\bl\bl -0.7071\bl\bl\bl 0.0278\bl\bl\bl 0.0162\\
\phantom{1}67.976\bl\bl\bl -0.6379\bl\bl\bl 0.0285\bl\bl\bl 0.0140\\
\phantom{1}62.742\bl\bl\bl -0.4937\bl\bl\bl 0.0408\bl\bl\bl 0.0125\\
----------------------------------------------------------------------\\
E =  715.61 MeV, E\_lo =  699.76 MeV, E\_hi =  731.45 MeV\\
Systematic = 0.0249, Preliminary = 0, CBELSA\_2014\_Hartmann\_PRL113-062001\\
163.335\bl\bl\bl -0.1684\bl\bl\bl 0.1025\bl\bl\bl 0.0038\\
151.045\bl\bl\bl -0.1793\bl\bl\bl 0.0599\bl\bl\bl 0.0053\\
142.373\bl\bl\bl -0.2888\bl\bl\bl 0.0447\bl\bl\bl 0.0078\\
135.072\bl\bl\bl -0.3406\bl\bl\bl 0.0414\bl\bl\bl 0.0081\\
128.682\bl\bl\bl -0.5064\bl\bl\bl 0.0369\bl\bl\bl 0.0130\\
122.820\bl\bl\bl -0.5747\bl\bl\bl 0.0335\bl\bl\bl 0.0136\\
117.258\bl\bl\bl -0.6330\bl\bl\bl 0.0405\bl\bl\bl 0.0199\\
112.024\bl\bl\bl -0.6199\bl\bl\bl 0.0360\bl\bl\bl 0.0147\\
106.978\bl\bl\bl -0.6730\bl\bl\bl 0.0335\bl\bl\bl 0.0148\\
102.005\bl\bl\bl -0.6214\bl\bl\bl 0.0312\bl\bl\bl 0.0149\\
\phantom{1}97.181\bl\bl\bl -0.6742\bl\bl\bl 0.0300\bl\bl\bl 0.0156\\
\phantom{1}92.407\bl\bl\bl -0.6291\bl\bl\bl 0.0289\bl\bl\bl 0.0149\\
\phantom{1}87.593\bl\bl\bl -0.6920\bl\bl\bl 0.0289\bl\bl\bl 0.0163\\
\phantom{1}82.819\bl\bl\bl -0.6456\bl\bl\bl 0.0301\bl\bl\bl 0.0152\\
\phantom{1}77.995\bl\bl\bl -0.5926\bl\bl\bl 0.0305\bl\bl\bl 0.0143\\
\phantom{1}73.022\bl\bl\bl -0.5031\bl\bl\bl 0.0321\bl\bl\bl 0.0117\\
\phantom{1}67.976\bl\bl\bl -0.4865\bl\bl\bl 0.0342\bl\bl\bl 0.0113\\
\phantom{1}62.742\bl\bl\bl -0.3185\bl\bl\bl 0.0396\bl\bl\bl 0.0073\\
\phantom{1}57.180\bl\bl\bl -0.2622\bl\bl\bl 0.0765\bl\bl\bl 0.0084\\
----------------------------------------------------------------------\\
\rm

The first header line carries energy information for the current dataset: The central (\tt E\rm)
photon beam energy in lab frame and the lower (\tt E\_lo\rm)
and upper (\tt E\_hi\rm) beam energy bounds of the current bin. Energies must be in MeV and the units
must be present in the header lines.
The second line gives a value for the relative systematic uncertainty (\tt Systematic\rm)
which should be an average number valid for all polar angle positions within this energy bin. This systematic uncertainty
is used for the observable rescaling option (see \tt FIX\_SCALES\rm\ configuration option). A value of
\tt Systematic = 0.05\rm\ would correspond to 5\% systematic uncertainty. 
The \tt Preliminary\rm\ field indicates whether this dataset is preliminary and/or unpublished and may not be used
for publications yet. Any data with \tt Preliminary = 1\rm\ will be ignored in fits if the configuration
option \tt USE\_PRELIMINARY\rm\ is not enabled.
The final entry in the second header line is a comment that can carry references or other identification
data for the current dataset. This comment may be up to 255 characters long and must not contain any spaces 
(use `\tt \_\rm'\ instead).

The following lines contain the actual observable data. Each line can hold up to four values which are\\
\tt\textit{theta angle\bl\bl\bl\bl observable value\bl\bl\bl\bl  statistical error\bl\bl\bl\bl}
[\textit{systematic error}]\rm\\
Polar angle positions have to be given in degrees, observable values\footnote{Asymmetry values must be within a range
of $[-1,+1]$.} for cross sections in $\mu$b/sr. Statistical and systematic
errors must be absolute values, e.g. in $\mu$b/sr for cross sections.
Systematic errors for each data point are optional and individual systematic uncertainties at each polar angle position
are not used by \PWA during the fit procedure.
Each energy bin entry must be terminated by a separator line like
`\tt------------------------------\rm' (do not use blank lines to separate entries).
Such a separator however may not be present at the very first line of an observable file, but must be on the end of the file
(terminating the last entry).
Polar angle positions within an energy bin may be in increasing or decreasing order (or no order at all).
Multiple energy bins within an observable file are possible, also here the ordering in energy is not important.
Experimental data files reside in the folders \tt data/ppi0\rm, \tt data/npip\rm, 
\tt data/peta\rm, ... (depending on the reaction $p \pi^0$, $n\pi^+$, $p\eta$, ...) of your \PWA installation.
You can add or create additional files in any folder, as data files will be referenced through the 
\tt PWA.cfg\rm\ file for your fit settings.

\subsection{Model multipoles}\label{sec_models}

\PWA needs model calculations for electromagnetic multipoles for use as start parameters and (if applicable) in penalty calculations.
These multipole files must reside in a directory \tt model\rm\ in your \PWA work directory, where
\tt model\rm\ may be a link to one of the existing model directories in the model repository of the \PWA package.
Typical multipole files look like the following:\\

\tt
$\bl$\bl W \bl\bl\bl\bl\bl\bl\bl\bl M1-(p pi0)\\
(MeV)\bl\bl\bl\bl\bl\bl Re\bl\bl\bl\bl\bl\bl\bl\bl\bl\bl Im\\
1074.0\bl\bl\bl -0.389350\bl\bl\bl-0.038572\\
1075.0\bl\bl\bl -0.427039\bl\bl\bl-0.042104\\
1076.0\bl\bl\bl -0.464727\bl\bl\bl-0.045635\\
1077.0\bl\bl\bl -0.502415\bl\bl\bl-0.049167\\
1078.0\bl\bl\bl -0.540103\bl\bl\bl-0.052699\\
1079.0\bl\bl\bl -0.577792\bl\bl\bl-0.056231\\
1080.0\bl\bl\bl -0.615480\bl\bl\bl-0.059763\\
1081.0\bl\bl\bl -0.653168\bl\bl\bl-0.063294\\
1082.0\bl\bl\bl -0.690856\bl\bl\bl-0.066826\\
1083.0\bl\bl\bl -0.728545\bl\bl\bl-0.070358\\
1084.0\bl\bl\bl -0.766233\bl\bl\bl-0.073890\\
\rm

\PWA ignores the first two lines of each multipole file, so these lines can carry column headers or further descriptions.
Starting with the third line the actual multipole information has to be present in the form\\
\tt\textit{center-of-mass energy\bl\bl\bl\bl real part\bl\bl\bl\bl imaginary part}\rm\\
where the center-of-mass energy $W$ is related to the lab photon beam energy $\omega$ according to
\begin{displaymath}
 W^2 = 2M\omega + M^2
\end{displaymath}
with the mass $M$ of the target nucleon. Real and imaginary parts of the multipole must be given in units of
$10^{-3}/m_{\pi^+}$. If any model provides multipoles in units of $10^{-3}\:\mathrm{fm}$ a conversion factor of
\begin{displaymath}
 \frac{m_{\pi^+}}{\hbar c} =\frac{139.5702\:\mathrm{MeV}}{197.3270\:\mathrm{MeV\cdot fm}}
\approx \sqrt\frac{1}{2}\:\mathrm{fm^{-1}}
\end{displaymath}
has to be applied before using these model calculations with \tt PWA\rm.
Multipole files have to follow the naming scheme
\tt E\textit{l}p.txt\rm,
\tt E\textit{l}m.txt\rm\ 
for electric multipoles, and
\tt M\textit{l}p.txt\rm,
\tt M\textit{l}m.txt\rm\ 
for magnetic multipoles, with angular momentum \tt\textit{l}\rm, and parity 
$+$ (\tt p\rm) or $-$ (\tt m\rm).
\PWA supports multipoles up to order $L = 10$.
The energy steps in multipole files can be arbitrary, however a rather fine step size ($\sim\hspace{-0.4em}1\:\mathrm{MeV}$)
is recommended in order to have reasonably precise multipole information for any SE fit energy available
(up to now, no interpolation on model values is performed).
For $d$ wave multipoles additional model values may be provided, carrying only Born term contributions or Bern terms as well as 
$\rho/\omega$ exchange mechanisms (see configuration option \tt D\_WAVES\rm).
These files must be named
\tt E\textit{2}p\_Born.txt\rm,
\tt E\textit{2}m\_Born.txt\rm,
\tt M\textit{2}p\_Born.txt\rm,
\tt M\textit{2}m\_Born.txt\rm,
and
\tt E\textit{2}p\_BornRhoOmega.txt\rm,
\tt E\textit{2}m\_BornRhoOmega.txt\rm,
\tt M\textit{2}p\_BornRhoOmega.txt\rm,
\tt M\textit{2}m\_BornRhoOmega.txt\rm,
respectively.
All filenames are case-sensitive.

Currently, for $p \pi^0$ the following models are provided: \vspace{-0.25em}\\
\begin{tabular}{ll}
\hspace{-0.5em}\tt model/ppi0/BG2011-01\rm & Bonn-Gatchina PWA (solution 2011-01) \cite{Model_BnGa} \vspace{-0.5em}\\
\hspace{-0.5em}\tt model/ppi0/BG2011-02\rm & Bonn-Gatchina PWA (solution 2011-02) \cite{Model_BnGa} \vspace{-0.5em}\\
\hspace{-0.5em}\tt model/ppi0/DMT\rm & DMT model \cite{Model_DMT} \vspace{-0.5em}\\
\hspace{-0.5em}\tt model/ppi0/MAID2007\rm & MAID 2007 \cite{Model_MAID} \vspace{-0.5em}\\
\hspace{-0.5em}\tt model/ppi0/SAID\rm & SAID CM12 \cite{Model_SAID} \vspace{-0.5em}\\
\hspace{-0.5em}\tt model/ppi0/ThrFit\rm & Empirical fit to $\pi^0$ threshold data \cite{PRL_pi0thres} \vspace{-0.5em}\\
\hspace{-0.5em}\tt model/ppi0/ThrChPT4\rm & $\chi$PT in 4$^\mathrm{th}$ order ($s$ and $p$ waves only)
\cite{Model_ChPT4}\vspace{-0.5em}\\
\hspace{-0.5em}\tt model/ppi0/ThrHBChPT4\rm & Heavy-baryon $\chi$PT in 4$^\mathrm{th}$ order ($s$ and $p$ waves only)
\cite{Model_ChPT4} \vspace{-0.5em}\\
\hspace{-0.5em}\tt model/ppi0/ThrHDT97\rm & Hanstein, Drechsel, Tiator solution ($s$ and $p$ waves only)
\cite{Model_HDT97}\vspace{-0.5em}\\
\hspace{-0.5em}\tt model/ppi0/ThrChMAID\rm & Chiral MAID \cite{Model_chiMAID} \vspace{-0.5em}\\
\hspace{-0.5em}\tt model/ppi0/ThrLG\rm & Lutz, Gasparyan calculation \cite{Model_LG} \vspace{-0.1em}\\
\end{tabular}

Any models prefixed with \tt Thr\rm\ are only supposed to be used within the $\pi$ threshold region up to
$\omega\sim 180\mathrm{MeV}$.

For $n \pi^+$ the following models are provided: \vspace{-0.25em}\\
\begin{tabular}{ll}
\hspace{-0.5em}\tt model/ppi0/BG2011-01\bl\rm & Bonn-Gatchina PWA (solution 2011-01) \cite{Model_BnGa} \vspace{-0.5em}\\
\hspace{-0.5em}\tt model/ppi0/BG2011-02\rm & Bonn-Gatchina PWA (solution 2011-02) \cite{Model_BnGa} \vspace{-0.5em}\\
\hspace{-0.5em}\tt model/ppi0/DMT\rm & DMT model \cite{Model_DMT} \vspace{-0.5em}\\
\hspace{-0.5em}\tt model/ppi0/MAID2007\rm & MAID 2007 \cite{Model_MAID} \vspace{-0.5em}\\
\hspace{-0.5em}\tt model/ppi0/SAID\rm & SAID CM12 \cite{Model_SAID} \vspace{-0.1em}\\
\end{tabular}

For $p \eta$ the following models are provided: \vspace{-0.25em}\\
\begin{tabular}{ll}
\hspace{-0.5em}\tt model/ppi0/BG2011-01\bl\rm & Bonn-Gatchina PWA (solution 2011-01) \cite{Model_BnGa} \vspace{-0.5em}\\
\hspace{-0.5em}\tt model/ppi0/BG2011-02\rm & Bonn-Gatchina PWA (solution 2011-02) \cite{Model_BnGa} \vspace{-0.5em}\\
\hspace{-0.5em}\tt model/ppi0/MAID2000\rm & $\eta$MAID 2000 \cite{Model_etaMAID} \vspace{-0.5em}\\
\hspace{-0.5em}\tt model/ppi0/MAID2014\rm & $\eta$MAID 2014 \vspace{-0.5em}\\
\hspace{-0.5em}\tt model/ppi0/SAID\rm & SAID \cite{Model_etaSAID} \vspace{-0.1em}\\
\end{tabular}

For $\Lambda K^+$ the following model is provided: \vspace{-0.25em}\\
\begin{tabular}{ll}
\hspace{-0.5em}\tt model/LambdaKp/RPR2011\rm & Gent RPR-2011 model\vspace{-0.1em}\\
\end{tabular}

\subsection{\PWA output files}\label{sec_output}

\PWA will create \tt .root\rm\ files of \tt TCanvas\rm\ objects with \tt TGraph\rm\ representations of fitted multipoles
and the model parameters used as start parameters during the fit. For each real and imaginary part of all multipoles up to
$L_\mathrm{max}$ individual files are created in the directories \tt plots.\textit{n}\rm\ for the different found solutions.
Files are named according to the scheme 
\tt ReE\textit{l}p.root\rm, \tt ImE\textit{l}p.root\rm,
\tt ReE\textit{l}m.root\rm, \tt ImE\textit{l}m.root\rm,
\tt ReM\textit{l}p.root\rm, \tt ImM\textit{l}p.root\rm,
\tt ReM\textit{l}m.root\rm, and \tt ImM\textit{l}m.root\rm\
for electric and magnetic multipoles of order \tt\textit{l}\rm\ and $+$ (\tt{p}\rm) or $-$ (\tt{m}\rm) parity.
These files can be displayed directly from within a \tt TBrowser\rm.
Further processing (e.g. generating plain text files with multipole results) is possible using the provided macros (see section
\ref{sec_macros}).

\section{Macros}\label{sec_macros}

\PWA offers different \tt ROOT\rm\ macros for further processing of fit results. This includes macros for generating plain text
files with multipole data as well as tools for drawing fitted multipoles and model calculations.

\subsection{Macro functions in  \tt macros/Extract.cpp\rm}

The file \tt Extract.cpp\rm\ provides a selection of different \tt ROOT\rm\ macros to plot multipole information
and create plain text files with SE fit results. These macros are made available by executing\\
\tt .L /home/\textit{username}/PWA/macros/Extract.cpp\rm\\
from a \tt ROOT\rm\ command line.

\begin{itemize}
\item
\tt Extract(Char\_t* REACT, Int\_t L\_MAX, Int\_t SOLUTIONS=1, Double\_t MASS\_INITIAL=938.2720)\rm

Converts \tt .root\rm\ files from \PWA output to plain text files. \tt Extract()\rm\ will
create text files for all multipoles up to multipole order \tt L\_MAX\rm\ from fit results in directories
\tt plots.0\rm\ up to \tt plots.$\lbrace$\textit{SOLUTIONS}-1$\rbrace$\rm.
\tt REACT\rm\ is an (arbitrary) reaction identifier that will be written in the header lines of output files,
e.g. \tt ''p pi0''\rm. \tt MASS\_INITIAL\rm\ (in MeV) is used for calculating the center-of-mass energy $W$. This parameter 
is optional, if it is omitted the proton mass is assumed.
Output files will be named
\tt E\textit{l}p.txt\rm,
\tt E\textit{l}m.txt\rm\ 
for electric multipoles, and
\tt M\textit{l}p.txt\rm,
\tt M\textit{l}m.txt\rm\ 
for magnetic multipoles, with angular momentum \tt\textit{l}\rm, and parity
$+$ (\tt p\rm) or $-$ (\tt m\rm). These files
will be created in the corresponding 
\tt plots.\textit{n}\rm\ directories and have the following format:\\

\tt
$\bl$\bl\bl W \bl\bl\bl\bl\bl\bl\bl\bl\bl\bl\bl\bl\bl\bl\bl E0+(p pi0)\\
\bl(MeV)\bl\bl\bl\bl\bl\bl Re\bl\bl\bl\bl\bl\bl DRe\bl\bl\bl\bl\bl\bl Im\bl\bl\bl\bl\bl\bl DIm\\
1074.018\bl\bl\bl -0.837\bl\bl\bl 0.107\bl\bl\bl-0.028\bl\bl\bl 0.195\\
1075.131\bl\bl\bl -0.793\bl\bl\bl 0.063\bl\bl\bl-0.253\bl\bl\bl 0.325\\
1076.242\bl\bl\bl -1.670\bl\bl\bl 0.801\bl\bl\bl-0.905\bl\bl\bl 1.106\\
1077.351\bl\bl\bl -2.188\bl\bl\bl 0.243\bl\bl\bl-0.112\bl\bl\bl 4.192\\
1078.460\bl\bl\bl -0.541\bl\bl\bl 0.027\bl\bl\bl\bl 0.408\bl\bl\bl 0.243\\
\rm

Each line consists of the center-of-mass energy $W$ of an SE fit as well as values and errors for real and imaginary
parts obtained from the fit. \textbf{Note:} After execution of \tt Extract()\rm\ it is recommended to restart
\tt ROOT\rm\ to avoid problems when drawing \tt TCanvas\rm\ objects. \tt Extract()\rm\ must be executed from your
work directory (having subdirectories \tt plots.\textit{n}\rm).

\item
\tt Multipole(Char\_t* Mlp, Int\_t SolLo=0, Int\_t SolHi=0, Bool\_t W=true,\\
\phantom{Multipole(}Bool\_t SAVE=false, Double\_t Lo=0.0, Double\_t Hi=0.0,\\
\phantom{Multipole(}Double\_t MASS\_INITIAL=938.2720, Int\_t D\_WAVES=MODEL)\rm

Plots the given multipole (real and imaginary parts) \tt Mlp\rm\ for solutions between \tt SolLo\rm\ and \tt SolHi\rm.
\tt Mlp\rm\ must be given in the form \tt ''E\textit{l}p''\rm,
\tt ''E\textit{l}m''\rm\ 
for electric multipoles, and
\tt ''M\textit{l}p''\rm,
\tt ''M\textit{l}m''\rm\ 
for magnetic multipoles, with angular momentum \tt\textit{l}\rm, and parity
$+$ (\tt p\rm) or $-$ (\tt m\rm). The plot will contain SE fit results as well as model predictions from the current
values in the directory \tt model\rm.
if \tt SAVE\rm\ is true, a \tt.pdf\rm\ file of the plot is created. \tt Lo\rm\ and \tt Hi\rm\ denote the plot range on the $y$-axis, 
if omitted or equal 0, the range is automatically adjusted.
\tt MASS\_INITIAL\rm\ (in MeV) is used for calculating the center-of-mass energy $W$. This parameter 
is optional, if it is omitted the proton mass is assumed.
\tt D\_WAVES\rm\ indicates, which $d$ wave contributions are used for plotting model multipoles (if supported by the current model).
\textbf{Note:} For successful operation of \tt Multipole()\rm\ it is necessary to have produced
fitted multipole text files with \tt Extract()\rm\ beforehand once. \tt Multipole()\rm\ must be executed from your
work directory (having subdirectories \tt plots.\textit{n}\rm\ and \tt model\rm).

\item
\tt Magnitude(Char\_t* Mlp, Int\_t SolLo=0, Int\_t SolHi=0, Bool\_t W=true,\\
\phantom{Magnitude(}Bool\_t SAVE=false, Double\_t Lo=0.0, Double\_t Hi=0.0,\\
\phantom{Magnitude(}Double\_t MASS\_INITIAL=938.2720, Int\_t D\_WAVES=MODEL)\rm

Plots the magnitude of the given multipole \tt Mlp\rm\ for solutions between \tt SolLo\rm\ and \tt SolHi\rm.
\tt Mlp\rm\ must be given in the form \tt ''E\textit{l}p''\rm,
\tt ''E\textit{l}m''\rm\ 
for electric multipoles, and
\tt ''M\textit{l}p''\rm,
\tt ''M\textit{l}m''\rm\ 
for magnetic multipoles, with angular momentum \tt\textit{l}\rm, and parity
$+$ (\tt p\rm) or $-$ (\tt m\rm). The plot will contain SE fit results as well as model predictions from the current
values in the directory \tt model\rm.
if \tt SAVE\rm\ is true, a PDF file of the plot is created. \tt Lo\rm\ and \tt Hi\rm\ denote the plot range on the $y$-axis, 
if omitted or equal 0, the range is automatically adjusted.
\tt MASS\_INITIAL\rm\ (in MeV) is used for calculating the center-of-mass energy $W$. This parameter 
is optional, if it is omitted the proton mass is assumed.
\tt D\_WAVES\rm\ indicates, which $d$ wave contributions are used for plotting model multipoles (if supported by the current model).
\textbf{Note:} For successful operation of \tt Magnitude()\rm\ it is necessary to have produced
fitted multipole text files with \tt Extract()\rm\ beforehand once. \tt Magnitude()\rm\ must be executed from your
work directory (having subdirectories \tt plots.\textit{n}\rm\ and \tt model\rm).

\item
\tt Phase(Char\_t* Mlp, Int\_t SolLo=0, Int\_t SolHi=0, Bool\_t W=true,\\
\phantom{Phase(}Bool\_t SAVE=false, Double\_t Lo=0.0, Double\_t Hi=0.0,\\
\phantom{Phase(}Double\_t MASS\_INITIAL=938.2720, Int\_t D\_WAVES=MODEL)\rm

Plots the phase (in radians) of the given multipole \tt Mlp\rm\ for solutions between \tt SolLo\rm\ and \tt SolHi\rm.
\tt Mlp\rm\ must be given in the form \tt ''E\textit{l}p''\rm,
\tt ''E\textit{l}m''\rm\ 
for electric multipoles, and
\tt ''M\textit{l}p''\rm,
\tt ''M\textit{l}m''\rm\ 
for magnetic multipoles, with angular momentum \tt\textit{l}\rm, and parity
$+$ (\tt p\rm) or $-$ (\tt m\rm). The plot will contain SE fit results as well as model predictions from the current
values in the directory \tt model\rm.
if \tt SAVE\rm\ is true, a PDF file of the plot is created. \tt Lo\rm\ and \tt Hi\rm\ denote the plot range on the $y$-axis, 
if omitted or equal 0, the range is automatically adjusted.
\tt MASS\_INITIAL\rm\ (in MeV) is used for calculating the center-of-mass energy $W$. This parameter 
is optional, if it is omitted the proton mass is assumed.
\tt D\_WAVES\rm\ indicates, which $d$ wave contributions are used for plotting model multipoles (if supported by the current model).
\textbf{Note:} For successful operation of \tt Phase()\rm\ it is necessary to have produced
fitted multipole text files with \tt Extract()\rm\ beforehand once. \tt Phase()\rm\ must be executed from your
work directory (having subdirectories \tt plots.\textit{n}\rm\ and \tt model\rm).

\item
\tt Model(Char\_t* Mlp, Bool\_t W=true, Bool\_t SAVE=false, Double\_t Lo=0.0, Double\_t Hi=0.0,\\
\phantom{Model(}Double\_t MASS\_INITIAL=938.2720, Int\_t D\_WAVES=MODEL)\rm

Plots model predictions for the given multipole (real and imaginary parts) \tt Mlp\rm.
\tt Mlp\rm\ must be given in the form \tt ''E\textit{l}p''\rm,
\tt ''E\textit{l}m''\rm\ 
for electric multipoles, and
\tt ''M\textit{l}p''\rm,
\tt ''M\textit{l}m''\rm\ 
for magnetic multipoles, with angular momentum \tt\textit{l}\rm, and parity
$+$ (\tt p\rm) or $-$ (\tt m\rm). The plot will contain only model predictions from the current
values in the directory \tt model\rm.
if \tt SAVE\rm\ is true, a PDF file of the plot is created. \tt Lo\rm\ and \tt Hi\rm\ denote the plot range on the $y$-axis, 
if omitted or equal 0, the range is automatically adjusted.
\tt MASS\_INITIAL\rm\ (in MeV) is used for calculating the center-of-mass energy $W$. This parameter 
is optional, if it is omitted the proton mass is assumed.
\tt D\_WAVES\rm\ indicates, which $d$ wave contributions are used for plotting model multipoles (if supported by the current model).
\textbf{Note:}  \tt Model()\rm\ must be executed from your
work directory (having a subdirectory \tt model\rm).


\item
\tt Chi2(Int\_t SOLUTION=0, Double\_t MASS\_INITIAL=938.2720)\rm

Draws the $\chi^2$ values for the given \tt SOLUTION\rm\ depending on the center-of-mass energy $W$.
\tt MASS\_INITIAL\rm\ (in MeV) is used for calculating $W$. This parameter 
is optional, if it is omitted the proton mass is assumed.
\tt Chi2()\rm\ must be executed from your
work directory (having subdirectories \tt plots.\textit{n}\rm).

\item
\tt Penalty(Int\_t SOLUTION=0, Double\_t MASS\_INITIAL=938.2720)\rm

Draws the penalty contributions for the given \tt SOLUTION\rm\ depending on the center-of-mass energy $W$.
\tt MASS\_INITIAL\rm\ (in MeV) is used for calculating $W$. This parameter 
is optional, if it is omitted the proton mass is assumed.
\tt Penalty()\rm\ must be executed from your
work directory (having subdirectories \tt plots.\textit{n}\rm).
\end{itemize}

\section{Isospin multipoles}

During the SE fits \PWA only determines multipole amplitudes for the chosen reaction (e.g. $A_{p \pi^0}$ or $A_{n \pi^+}$).
For reactions on an initial proton these charge channel multipoles are related to isospin
$I = {}_p\frac{1}{2}$ and $I = \frac{3}{2}$ multipoles according to
\begin{displaymath}
\begin{array}{lll}
A_{p \pi^0} &=& \phantom{\sqrt{2}}\hspace{0.75em}A_p^{1/2} + \frac{2}{3} A^{3/2}\\
A_{n \pi^+} &=& \sqrt{2}\left(A_p^{1/2} - \frac{1}{3} A^{3/2}\right)
\end{array}
\end{displaymath}
or the inverted relations
\begin{displaymath}
\begin{array}{lll}
A_p^{1/2} &=& \frac{1}{3}\left(\sqrt{2}A_{n \pi^+} + \phantom{\frac{1}{\sqrt{2}}} A_{p \pi^0}\right)\\
A^{3/2} &=& \phantom{\frac{1}{3}\sqrt{2}}\hspace{0.77em} A_{p \pi^0} \hspace{0.24em}- \frac{1}{\sqrt{2}} A_{n \pi^+}
\end{array}
\end{displaymath}
The directory \tt isospin\rm\ within the main of the \PWA package contains some tools to create isospin multipoles
from charge channel multipole fit results.

\subsection{\tt isospin\rm\ directory contents}

The \tt isospin\rm\ directory (which acts as some sort of `isospin working directory') contains several subdirectories
\begin{itemize}
\item
\tt data/ppi0\rm\ and \tt data/npip\rm\\
These directories hold SE multipole results obtained from \PWA fits on $p\pi^0$ and $n\pi^+$ data. These multipoles
must have the text format created by the \tt Extract()\rm\ function and follow the naming scheme
\tt E\textit{l}p.txt\rm,
\tt E\textit{l}m.txt\rm\ 
for electric multipoles, and
\tt M\textit{l}p.txt\rm,
\tt M\textit{l}m.txt\rm\ 
for magnetic multipoles, with angular momentum \tt\textit{l}\rm, and parity
$+$ (\tt p\rm) or $-$ (\tt m\rm).
So, basically the \tt data/ppi0\rm\ and \tt data/npip\rm\ directories hold copies of the multipole data created
by \tt Extract()\rm\ in the \tt plots.\textit{n}\rm\ directories.
\item
\tt model/ppi0\rm\ and \tt model/npip\rm\\
These directories hold files for theoretical model multipoles, that are identical to the file format described in section
\ref{sec_models}. In fact it is recommended to organise \tt model/ppi0\rm\ and \tt model/npip\rm\ as
symbolic links to the desired model, e.g. use\\
\tt ln -s /home/\textit{username}/PWA/model/ppi0/MAID/ /home/\textit{username}/PWA/isospin/model/ppi0\rm\\
\tt ln -s /home/\textit{username}/PWA/model/npip/MAID/ /home/\textit{username}/PWA/isospin/model/npip\rm\\
Model values for charge channel multipoles are only used for plotting multipoles, extraction
of isospin multipoles uses the analytic relations above and is model-independent.
\item 
\tt isospin\rm\\
This directory will hold text files for isospin multipoles determined from $p\pi^0$ and $n\pi^+$ charge
channel multipoles. These isospin multipole files follow the naming scheme
\tt E\textit{l}p\_p12.txt\rm,
\tt E\textit{l}m\_p12.txt\rm\ 
for electric $_p\frac{1}{2}$ multipoles,
\tt E\textit{l}p\_32.txt\rm,
\tt E\textit{l}m\_32.txt\rm\ 
for electric $\frac{3}{2}$ multipoles,
\tt M\textit{l}p\_p12.txt\rm,
\tt M\textit{l}m\_p12.txt\rm\ 
for magnetic $_p\frac{1}{2}$ multipoles, and
\tt M\textit{l}p\_32.txt\rm,
\tt M\textit{l}m\_32.txt\rm\ 
for magnetic $\frac{3}{2}$ multipoles,
all with angular momentum \tt\textit{l}\rm, and parity
$+$ (\tt p\rm) or $-$ (\tt m\rm).
The isospin multipole file format is identical to the charge channel multipole format created by \tt Extract()\rm\
and contains a two-line header and multipole data formed by
center-of-mass energy $W$ as well as values and errors for real and imaginary parts.

\end{itemize}

\subsection{Isospin multipole macros in \tt isospin/macros/IsoExtract.cpp\rm}

The file \tt IsoExtract.cpp\rm\ provides a selection of different \tt ROOT\rm\ macros to create and plot isospin
multipoles. These macros are made available by executing\\
\tt .L /home/\textit{username}/PWA/isospin/macros/IsoExtract.cpp\rm\\
from a \tt ROOT\rm\ command line.

\begin{itemize}
\item
\tt IsoExtract(Int\_t L\_MAX = 5, Double\_t W\_LO=0.0, Double\_t W\_HI=2500.0)\rm

Creates isospin $I = {}_p\frac{1}{2}$ and $I = \frac{3}{2}$ multipoles from $p \pi^0$ and
$n \pi^+$ charge channel multipoles up to order \tt L\_MAX\rm\ for an center-of-mass energy range
between \tt W\_LO\rm\ and \tt W\_HI\rm. Parameters are optional, if omitted the given defaults are used.
Charge channel multipoles up to \tt L\_MAX\rm\ \textbf{must} be present in the 
\tt data/ppi0\rm\ and \tt data/npip\rm\ directories. Output multipole tables will be written to the 
\tt isospin\rm\ subdirectory.
\tt IsoExtract()\rm\ will perform multipole calculations on either $p \pi^0$ or $n \pi^+$
SE fit energies, depending on which channel has more energy points (typically this will be $p \pi^0$).
If $p \pi^0$ and $n \pi^+$ energies do not match, the closest energy position from the corresponding 
channel is used without applying any interpolation scheme.

\item
\tt IsoMultipole(Char\_t* Mlp, Char\_t* Iso, Bool\_t SAVE=false,\\
\phantom{IsoMultipole(}Double\_t Lo=0.0, Double\_t Hi=0.0)\rm

Plots the given isospin multipole (real and imaginary parts) \tt Mlp\rm.
\tt Mlp\rm\ must be given in the form \tt ''E\textit{l}p''\rm,
\tt ''E\textit{l}m''\rm\ 
for electric multipoles, and
\tt ''M\textit{l}p''\rm,
\tt ''M\textit{l}m''\rm\ 
for magnetic multipoles, with angular momentum \tt\textit{l}\rm, and parity
$+$ (\tt p\rm) or $-$ (\tt m\rm). \tt Iso\rm\ must be given as \tt ''p12''\rm\ and \tt ''32''\rm\
for isospin $I = {}_p\frac{1}{2}$ and $I = \frac{3}{2}$.
The plot will contain fit results as well as model predictions from the current
values in the directory \tt model\rm.
if \tt SAVE\rm\ is true, a \tt.pdf\rm\ file of the plot is created. \tt Lo\rm\ and \tt Hi\rm\ denote the plot range on the $y$-axis, 
if omitted or equal 0, the range is automatically adjusted.
\textbf{Note:} For successful operation of \tt IsoMultipole()\rm\ it is necessary to have produced
fitted multipole text files with \tt IsoExtract()\rm\ beforehand once. \tt IsoMultipole()\rm\ must be executed from your
work directory (having subdirectories \tt data\rm\ and \tt model\rm).
\end{itemize}

\section{Datasets included in \PWA package}

The \PWA package comes with a selection of different experimental datasets for cross sections and polarisation observables
for $\pi^0$, $\pi^+$, and $\eta$ photoproduction. These datasets consist of recent Mainz, Bonn, and JLab measurements as well as
some additional older results that are also present in the SAID database.

\subsection{$\pi^0$ data}
Datasets for $\pi^0$ photoproduction observables can be found in the directory 
\tt data/ppi0\rm\ of the \PWA package. The following datasets for cross sections are provided:
\begin{itemize}
\item[$\sigma_0$:\:]
\tt sg0\_CBMAMI\_2008\_Prakhov\_PRL111-062004.txt\rm \\
\tt sg0\_CBMAMI\_2014\_Prakhov.txt\rm

\item[$\sigma_T$:\:]
\tt sgT\_CBMAMI\_2014\_Schumann-Otte.txt\rm

\item[$\sigma_E$:\:]
\tt sgE\_DAPHNE\_2001\_Preobrajenski.txt\rm

\item[$\sigma_F$:\:]
\tt sgF\_CBMAMI\_2014\_Schumann-Otte.txt\rm
\end{itemize}
The following datasets on polarisation asymmetries are provided:
\begin{itemize}
\item[$\Sigma$:\:]
\tt S\_CBMAMI\_2008\_Prakhov\_PRL111-062004.txt\rm \\
\tt S\_TAPSA2\_2001\_Leukel.txt\rm \\
\tt S\_GRAAL\_2005\_Bartalini\_EPJA26-399.txt\rm

\item[$T$:\:]
\tt T\_CBMAMI\_2014\_Kashevarov.txt\rm \\
\tt T\_CBELSA\_2014\_Hartmann\_PRL113-062001.txt\rm

\item[$P$:\:]
\tt P\_CBELSA\_2014\_Hartmann\_PRL113-062001.txt\rm \\
\tt P\_DNPL\_1972\_Prentice\_NPB41-353.txt\rm \\
\tt P\_DNPL\_1979\_Bussey\_NPB154-492.txt\rm \\
\tt P\_KHARKOV\_1976\_Derebchinski\_JETP43-218.txt\rm \\
\tt P\_KHARKOV\_1978\_Zybalov\_SJNP28-52.txt\rm \\
\tt P\_KHARKOV\_1980\_Bratashevski\_NPB166-525.txt\rm \\
\tt P\_KHARKOV\_1983\_Belyaev\_NPB213-201.txt\rm

\item[$E$:\:]
\tt E\_CBELSA\_2014\_Gottschall\_PRL112-012003.txt\rm

\item[$F$:\:]
\tt F\_CBMAMI\_2014\_Kashevarov.txt\rm

\item[$G$:\:]
\tt G\_DAPHNE\_2005\_Ahrens\_EJPA26-135.txt\rm \\
\tt G\_CBELSA\_2012\_Thiel\_PRL109-102001.txt\rm

\item[$H$:\:]
\tt H\_CBELSA\_2014\_Hartmann\_PRL113-062001.txt\rm
\end{itemize}

\subsection{$\pi^+$ data}
Datasets for $\pi^+$ photoproduction observables can be found in the directory 
\tt data/npip\rm\ of the \PWA package. The following datasets for cross sections are provided:
\begin{itemize}
\item[$\sigma_0$:\:]
\tt sg0\_JLAB\_2009\_Dugger\_PRC79-065206.txt\rm \\
\tt sg0\_DAPHNE\_2004\_Ahrens\_EPJA21-323.txt\rm \\
\tt sg0\_DAPHNE\_2000\_Beck\_PRC61-034604.txt\rm \\
\tt sg0\_MAMI\_1999\_Branford\_PRC61-014603.txt\rm \\
\tt sg0\_SAL\_1996\_Fisum\_PRC53-1278.txt\rm \\
\tt sg0\_SAL\_1999\_Korkmaz\_PRL83-3609.txt\rm \\
\tt sg0\_PHOENICS\_1994\_Buechler\_NPA570-580.txt\rm \\
\tt sg0\_TOKYO\_1979\_Fujii\_NPB120-395.txt\rm \\
\tt sg0\_BONN\_1972\_Fischer\_ZP253-38.txt\rm \\
\tt sg0\_ORSAY\_1968\_Betourne\_PR172-1343.txt\rm \\
\tt sg0\_ORSAY\_1962\_Walker\_PRL8-37.txt\rm \\
\tt sg0\_CALTECH\_1967\_Ecklund\_PR159-1195.txt\rm \\
\tt sg0\_GLASGOW\_1964\_Leith\_PL8-355.txt\rm

\item[$\sigma_E$:\:]
\tt sgE\_DAPHNE\_2001\_Preobrajenski.txt\rm
\end{itemize}
The following datasets on polarisation asymmetries are provided:
\begin{itemize}
\item[$\Sigma$:\:]
\tt S\_JLAB\_2013\_Dugger\_PRC88-065203.txt\rm \\
\tt S\_DAPHNE\_2000\_Beck\_PRC61-034604.txt\rm \\
\tt S\_GRAAL\_2000\_Ajaka\_PLB475-372.txt\rm \\
\tt S\_GRAAL\_2002\_Bartalini\_PLB544-113.txt\rm \\
\tt S\_LEGS\_2001\_Blanpied\_PRC64-025203.txt\rm \\
\tt S\_KHARKOV\_1989\_Getman\_VANT8-3.txt\rm \\
\tt S\_KHARKOV\_1976\_Ganenko\_SJNP23-52.txt\rm \\
\tt S\_DNPL\_1979\_Bussey\_NPB154-205.txt\rm \\
\tt S\_STANFORD\_1963\_Smith\_PR130-2429.txt\rm

\item[$T$:\:]
\tt T\_PHOENICS\_1996\_Dutz\_NPA601-319.txt\rm \\
\tt T\_KHARKOV\_1981\_Getman\_NPB188-397.txt\rm \\
\tt T\_DNPL\_1979\_Bussey\_NPB154-205.txt\rm

\item[$P$:\:]
\tt P\_KHARKOV\_1981\_Getman\_NPB188-397.txt\rm \\
\tt P\_DNPL\_1979\_Bussey\_NPB154-205.txt\rm

\item[$G$:\:]
\tt G\_DAPHNE\_2005\_Ahrens\_EJPA26-135.txt\rm \\
\tt G\_KHARKOV\_1984\_Belyaev\_SJNP40-83.txt\rm \\
\tt G\_DNPL\_1980\_Bussey\_NPB169-403.txt\rm

\item[$H$:\:]
\tt H\_KHARKOV\_1986\_Belyaev\_SNJP43-947.txt\rm \\
\tt H\_DNPL\_1980\_Bussey\_NPB169-403.txt\rm
\end{itemize}

\subsection{$\eta$ data}
Datasets for $\eta$ photoproduction observables can be found in the directory 
\tt data/peta\rm\ of the \PWA package. The following datasets for cross sections are provided:
\begin{itemize}
\item[$\sigma_0$:\:]
\tt sg0\_CBMAMI\_2010\_McNicoll\_PRC82-035208.txt\rm
\end{itemize}
The following datasets on polarisation asymmetries are provided:
\begin{itemize}
\item[$\Sigma$:\:]
\tt S\_GRAAL\_2007\_Bartalini\_EPJA33-169.txt\rm

\item[$T$:\:]
\tt T\_CBMAMI\_2014\_Kashevarov\_PRL113-102001.txt\rm

\item[$F$:\:]
\tt F\_CBMAMI\_2014\_Kashevarov\_PRL113-102001.txt\rm
\end{itemize}


\begin{thebibliography}{00}

\bibliographystyle{unsrt}

\bibitem{PRL_pi0thres}
D. Hornidge et al.,
Phys. Rev. Lett. \textbf{111}, 062004 (2013)

\bibitem{Model_BnGa}
\tt http://pwa.hiskp.uni-bonn.de/baryon\_x.htm\rm

\bibitem{Model_DMT}
\tt http://portal.kph.uni-mainz.de/MAID/dmt/dmt2001.html\rm

\bibitem{Model_MAID}
\tt http://portal.kph.uni-mainz.de/MAID/maid2007/maid2007.html\rm

\bibitem{Model_SAID}
\tt http://gwdac.phys.gwu.edu/analysis/pr\_analysis.html\rm

\bibitem{Model_ChPT4}
M. Hilt, S. Scherer, L. Tiator,
arXiv:1301.5576 [nucl-th]

\bibitem{Model_HDT97}
O. Hanstein, D. Drechsel, L. Tiator,
Phys. Lett. B \textbf{399}, 13 (1997)

\bibitem{Model_LG}
A. Gasparyan, M.F.M. Lutz,
Nucl. Phys. A \textbf{848}, 126 (2010)

\bibitem{Model_chiMAID}
\tt http://portal.kph.uni-mainz.de/MAID//chiralmaid/\rm

\bibitem{Model_etaMAID}
\tt http://portal.kph.uni-mainz.de/MAID/eta/etamaid.html\rm

\bibitem{Model_etaSAID}
\tt http://gwdac.phys.gwu.edu/analysis/pre\_analysis.html\rm

\end{thebibliography}

\end{document}

